\documentclass{article}

\usepackage{amsmath}
\usepackage{amsfonts}


\title{CS 2051 Practice Exam 1}
\author {TAs: Alan Tao, Nithya Jayakumar, Rishi Raman}
\date{Spring 2022}

\begin{document}
    \maketitle
    \section{Logic}
        Let p, q, and r be the propositions. \\
    p: Grizzly bears have been seen in the area. \\
    q: Hiking is safe on the trail. \\
    r: Berries are ripe along the trail. \\
    Write these propositions using p, q, and r and logical connectives.
    \begin{enumerate}
        \item Berries are ripe along the trail, but grizzly bears have not been seen in the area.
        \item If berries are ripe along the trail, hiking is safe if and only if grizzly bears have not been seen in the area.
        \item For hiking on the trail to be safe, it is necessary but not sufficient that berries not be ripe along the trail and for grizzly bears not to have been seen in the area.
        \item Hiking is not safe on the trail whenever grizzly bears have been seen in the area and berries are ripe along the trail.
    \end{enumerate}
    \hrule
    \vspace{5 pt}
    For a compound proposition combining n propositions, how many rows will the truth table have?
    \vspace{5 pt}
    \hrule
    \vspace{5 pt}
    Give the negation for each of the following types of nested quantifiers.
    \begin{enumerate}
        \item $\forall x \forall y P(x,y)$
        \item $\forall x \exists y P(x,y)$
        \item $\exists x \forall y P(x,y)$
        \item $\exists x \exists y P(x,y)$
    \end{enumerate}
    \vspace{5 pt}
    \hrule
    \vspace{5 pt}
        Let L(x, y) be the statement “x loves y,” where the domain for both x and y consists of all people in the world. Use quantifiers to express each of these statements.
        \begin{enumerate}
            \item Everybody loves Jerry.
            \item Everybody loves somebody.
            \item There is somebody whom everybody loves.
            \item Nobody loves everybody.
            \item There is somebody whom Lydia does not love.
            \item There is somebody whom no one loves.
            \item There is exactly one person whom everybody loves.
            \item There are exactly two people whom Lynn loves.
            \item Everyone loves himself or herself.
            \item There is someone who loves no one besides himself or herself.
        \end{enumerate}
    \section{Proof Techniques}
		Prove the following using either the method of Direct Proof, Proof by Contraposition, Proof by Contradiction, or Proof by Cases:
	\begin{enumerate}
		
		\item Suppose $a$ and $b$ are integers. If $a \lvert b$, then $a \lvert (3b^{3}-b^{2} +5b)$.
		\item If $p$ is prime, and $k$ is an integer for which $0 < k < p$, then $p$ divides $\binom n k$.
		\item If $x \in \mathbb{R}$ and $0 < x < 4$, then $\frac{4}{x(4 - x)} \ge 1$.
		\item Every odd integer is a difference of two squares.
		\item If $5 \lvert xy$, then $5\lvert x$ and $5 \lvert y$.

		For reference: $a\equiv b \mathrm{(mod }m\mathrm{)} \leftrightarrow \exists k \in \mathbb{Z} \mathrm{s.t.} a - b = km$. In other words, $a \equiv b$ means $a$ and $b$ have the same remainders when divided by another number, say, $m$.

		\item Let $a, b\in \mathbb{Z}$, and $n\in\mathbb{N}$. If $a \equiv b \mathrm{(mod }n \mathrm{)}$, then $a^{2} \equiv ab \mathrm{(mod }n\mathrm{)}$.

		\item If $a \in \mathbb{Z}$, and $a \equiv 1 \mathrm{(mod } 5 \mathrm{)}$, then $a^{2} \equiv 1 \mathrm{(mod }5\mathrm{)}$.

		\item If $a \equiv b \mathrm{(mod }n\mathrm{)}$ and $c \equiv d \mathrm{(mod }n\mathrm{)}$, then $ac \equiv bd \mathrm{(mod }n\mathrm{)}$.	

		\item If $n$ is odd, then $8 \lvert (n^{2} - 1)$.

		\item Let $n \in \mathbb{N}$. If $2^{n} - 1$ is prime, then $n$ is prime.
		
		\item If $A$ and $B$ are sets, then $A \cap (B - A) = \emptyset$

		\item Suppose $a, b \in \mathbb{Z}$. If $4 \lvert (a^{2} + b^{2})$, then $a$ and $b$ are not both odd.

		\item Suppose $a, b, c \in \mathbb{Z}$. If $a^{2} + b^{2}= c^{2}$, then  $a$ or $b$ is even.

		\item For every positive $x \in \mathbb{Q}$, there exists a positive $y \in \mathbb{Q}$ such that $y < x$.

		\item Suppose $a, b \in \mathbb{R}$. If $a$ is rational and $ab$ is irrational then $b$ is irrational.
	\end{enumerate}
    \section{Set theory}
        \begin{enumerate}
            \item Let $S = \{\{1\}, \{2, 3\}, 4, 5, \{\{6\}\}\}.$ Which of the following is a subset of $S$?
                \begin{itemize}
                    \item[A.] $\{1, 2, 3\}$
                    \item[B.] $\{\{\{6\}\}\}$
                    \item[C.] $\{2, 3\}$
                    \item[D.] $\{1\}$
                \end{itemize}
            \item What is the power set of $\{\emptyset, \{\emptyset\}\}$?
            \item If $A$ and $B$ are sets with the same power set, can you conclude that $A = B$?
            \item Let $A = \{1, 2, 3\}$, $B = \{4, 5\}$, and $C = \{6, 7\}$. Find $C \times A \times B.$
            \item Let $A = \{1, 2, 3, 4, 5\}$ and $B = \{1, 2, 3, 4, 5, 6, 7, 8\}$. Find $A \cup B$, $A \cap B$, $A - B$, and $B - A.$
            \item Let $A$ and $B$ be sets. Show that $(A \cap B) \subseteq A$. 
            \item Let $A$ and $B$ be sets. Prove that $\overline{A \cap B} = \overline{A} \cup \overline{B}$.
            \item What is the cardinality of the set $\{\emptyset, \{\emptyset\}, \{\emptyset, \{\emptyset\}\}\}$?
            \item Is $\mathbb{Z}$ countable? Why?
        \end{enumerate}
    \section{Functions}
        \begin{enumerate}
            \item Give an example of a function from $\mathbb{N}$ to $\mathbb{N}$ that is:
                \begin{itemize}
                    \item one-to-one but not onto
                    \item onto but not one-to-one
                    \item one-to-one and onto
                    \item neither one-to-one nor onto
                \end{itemize}
            \item Is the function $f(x) = \frac{x^2 + 1}{x^2 + 2}$ a bijection from $\mathbb{R}$ to $\mathbb{R}$?
        \end{enumerate}
    \section{Relations}
	    Show that this relation for any $m > 1$ is an equivalence relation. \\
    $R = \{(a, b) \lvert a \equiv b (\bmod m)\}$. Number theory has not been talked about yet, but use the fact that $a \equiv b (\bmod m)$ if and only if $m$ divides $a - b$.
    \vspace{5 pt}
    \hrule
    \vspace{5 pt}
    Prove the following theorem: \\
    Let R be an equivalence relation on a set A. These statements for elements a and b of A are equivalent: \\
    \begin{enumerate}
        \item $aRb$
        \item $[a] = [b]$
        \item $[a] \cap [b] \neq \emptyset$
    \end{enumerate}
    \vspace{5 pt}
    \hrule
    \vspace{5 pt}
    Determine whether the relation R on the set of all integers is reflexive, symmetric, antisymmetric, and/or transitive, where $xRy$ if and only if:
    \begin{enumerate}
        \item $x \neq y$.
        \item $xy \geq 1$.
        \item $x = y + 1$ or $x = y - 1$.
        \item $x = y^2$
        \item $x$ is a multiple of y.
    \end{enumerate}
    \section{Induction}
	\begin{enumerate}
	\item Concerning the Fibonacci sequence, prove that $F_{1} + F_{2} + F_{3} + ... + F_{n} = F_{n + 2} - 1$.
	\item If $n \in \mathbb{N}$, then $\frac{1}{1\cdot 2} + \frac{1}{2\cdot 3} + \frac{1}{3\cdot 4} + \frac{1}{4\cdot 5} + ... + \frac{1}{n(n+1)} = 1 - \frac{1}{n + 1}$.
	\item Prove that $2^{n} + 1 \le 3^{n}$ for every positive integer $n$.
	\item Prove that $9\lvert (4^{3n} + 8)$ for every integer $n \ge 0$.
	\item If $n \in \mathbb{N}$, then $2^{1} + 2^{2} + 2^{3} + ... + 2^{n} = 2^{n + 1} - 2$.
	\end{enumerate}	
\end{document}
