\documentclass{article}
\usepackage[margin=1in]{geometry}
\usepackage{amsmath, amssymb, amsthm}
\usepackage{enumitem}

%Cases Environment
\newlist{Cases}{enumerate}{3}
\setlist[Cases]{leftmargin = .25in, label = {Case \arabic*.}, topsep = 0.01in, itemsep = 0.04in, itemindent = .5in, parsep = 0in}

%Formatting and Spacing
\setitemize[1]{noitemsep, parsep = 5pt, topsep = 5pt}
\setenumerate[1]{label = (\alph*), parsep = 1pt, topsep = 5pt}
\setlength\parindent{0pt}
\linespread{1.1}

%Custom Title Fields
\newcommand{\lectTitle}{Lecture 11 Notes}
\newcommand{\lectTime}{February 16, 2022}
\newcommand{\lectClass}{Honors Discrete Mathematics}
\newcommand{\lectClassInstructor}{Gerandy Brito}
\newcommand{\lectSection}{Spring 2022}
\newcommand{\lectAuthorName}{Sarthak Mohanty}

%Headers and Footers
\usepackage{fancyhdr}
\usepackage{extramarks}
\pagestyle{fancy}
\lhead{\lectTime}
\chead{\lectClass \ (\lectClassInstructor)}
\rhead{\lectTitle}
\cfoot{\thepage}
\renewcommand\headrulewidth{0.4pt}
\renewcommand\footrulewidth{0.4pt}

\title{
    \vspace{2in}
    \textbf{\lectClass:\\ \lectTitle}\\
    \vspace{0.1in}\large{\textit{\lectClassInstructor\ \lectSection}}
    \vspace{3in}
    \author{\textbf{\lectAuthorName}}
    \date{}
}

\begin{document}

\maketitle
\pagebreak

\section*{Ordering Relations}
    Recall that a relation is (partially) ordered if it is reflexive, antisymmetric, and transitive. 
    
    Examples:
    \begin{itemize}
        \item $(\mathbb{N}, \le)$
        \item $(\mathbb{N}, \setminus)$ 
        \item $([n], \le)$
        \item $([n], \setminus)$
        \item $(\mathcal{P}(S), \subseteq)$
        \item $(\prod_{n}, \le)$
    \end{itemize}
    
    Principle of Well-ordered Induction
    \begin{itemize}
        \item Totally ordered: every pair of elements are comparable. 
        \item Well ordered: every subset has a least element.
    \end{itemize}
    
    \textbf{Definition}: An element $a \in (A, \le)$ is called minimal (maximal) if there is \underline{no} element $x$ such that $x \le a$. ($a \le x)$.
    
    \textbf{Example} Let the ordered relation $([6], \setminus)$. 
    
    We can create a partial ordering of the this relation using a Hasse diagram.
    
    level one: 1
    
    level two: 2, 3, 5
    
    level three: 4, 6
    
    1 connected to 2, 3, 5
    
    2 is connected to 4, 6
    
    3 is connected to 6
    
    There are three maximal elements: 4, 5, 6.
    
    \vspace{1.5mm}
    \textbf{Lemma}: If a poset $(A, \le)$ is finite then there is at least one minimal element. If $(A, \le)$ is totally orderd, then minimal element is unique!
    
    \vspace{1.5mm}
    \textbf{Proof}: Take any $a \in A$.
    \begin{itemize}
        \item If $a$ is minimal, then the statement is complete
        \item Else, there is $a_{1} \le a$. Repeat for $a_{1}$ until we have found our desired element (we know this algorithm terminates because $A$ is finite) and outputs a minimal element.
    \end{itemize}
    
    \vspace{1.5mm}
    \textbf{Example}
    Let $(\mathcal{P}(\{1, 2, 3, 4\}), \subseteq)$. Then the Hasse Diagram is 
    Level 1: $\emptyset$
    
    Level 2: $\{1\}, \{2\}, \{3\}, \{4\}$.
    
    Level 3: $\{1, 2\}, \{1, 3\}, \{2, 3\}, \{2, 4\}, \{3, 4\}, \{1, 4\}$
    
    dots 
    up to the maximal element $\{1, 2, 3, 4\}$
    
    \vspace{1.5mm}
    \textbf{Definition} Given a poset $(A \le)$ and $S \subseteq A$, a \textit{least upper bound} for $S$ is an element $y$ such that 
    \begin{itemize}
        \item $s \le y \forall s \in S$.
        \item If $x$ is an upper bound for $S$, $y \le x$.
    \end{itemize}
    Analogously we can define a greatest lower bound.
    
    \vspace{1.5mm}
    \textbf{Example}: 
    
    Level One: 1
    
    Level 2: 2, 3, 5, 7
    
    Level 3: 4, 6, 9, 10
    
    Level 4: 8
    
    1 connected to 2, 3, 5, 7
    
    2 connected to 4, 6, 9, 10
    
    3 connected to 6, 9
    
    5 connected to 10
    
    4 connected to 6
    
    Then $6$ is the least upper bound
    
    \vspace{1.5mm}
    Hence $\{1, 2, 3\}$ lub is 6
    
    $\{1, 2, 4\}$ Then the l.u.b is 4
    
    For $\{3, 5, 7\}$, then l.u.b DNE
    
    \vspace{1.5mm}
    \textbf{Lemma} The l.u.b is unique
    
    \vspace{1.5mm}
    \textbf{Lemma} Let $x, y$ be l.u.b for $S$
    \begin{itemize}
        \item $x$ is an upper bound is l.u.b so $y \le x$.
        \item $y$ is an upper bound and $x$ is l.u.b so $x \le y$.
        \item By antisymmetric property $x = y$
    \end{itemize}
    
    \vspace{1.5mm}
    \textbf{Definition} A poset is called a \textit{lattice} if very pair of elements has a l.u.b.\ and g.l.b.
    
    \vspace{1.5mm}
    \textbf{Example} $(\mathbb{N}, \setminus)$ is a lattice
    
    $a, b \in \mathbb{N}$
    
    l.u.b.\ of $(a, b)  = $
    
    Hasse Diagram
    
    Level one: d
    
    Level two: a, b
    
    Level three: c
    
    $d$ connected to $a, b$
    
    $a, b$ connected to $c$.
    
    $a \setminus c$, $b \setminus c$, $d \setminus a$, $d \setminus b$.
    
    Note that $c$ is the mminimal common multiple.
    
    Note that $d$ is the greatest common divisor
    3
    
    \vspace{1.5mm}
    
    Let the set of lists equal $\{CS1, CS2, \dots, CSn\}$.
    
    $a \mathcal{R} b$ if you must go through class $a$ before class $b$.
    
    Hasse Diagram
    
    Level One: CS1
    
    Level Two: CS3, CS4, CS5
    
    Level Three: CS2
    
    CS1 is connected to CS3, CS4, CS5
    
    CS3 is connected to CS2
    
    CS4 is connected to CS2
    
    \vspace{1.5mm}
    \textbf{Definition}: Given a poset $(A, \le)$. A \underline{topological sorting} of $A$ is an enumeration of elements in $A$: $a_{1}, a_{2}, a_{3}, \dots, a_{n}$ such that if $a_{i} \le a_{j}$ then $i \le j$.
    
    Can we always find a topological sorting for a finite set $A$? Answer: yes; find the least element, exclude the element from the set, and find the least element of that subset. Rinse and repeat until you have found what you are looking for.
    
    In other words, to find a topological sorting on the poset $(S, \subseteq)$
    
    \begin{enumerate}
        \item Find a minimal element of $S$, $s_{1}$,
        \item Iterate on $S \setminus \{s_{1}\}$
    \end{enumerate}
    
    \vspace{1.5mm}
\textbf{Example}: $(\{1, 2, 4, 5, 10, 20\}, \setminus)$.

Hasse Diagram: 
    Level 1: 1
    
    Level 2: 2, 5
    
    Level 3: 4, 10
    
    Level 4: 20
    
    1 is connected to 2, 5
    
    2 is connected to 4
    
    5 is connected to 10
    
    4 is connected to 20
    
    10 is connected to 20
    
    A topological sorting is given by $1, 5, 2, 10, 4, 20$. However, there are multiple topological sortings, can you find one more?

    Question: How many topological sortings exist for a totally ordered set.
    
    \vspace{1.5mm}
    \textbf{Proposition}: Let $(A, \le)$ be a poset s.t.\ $A$ has a unique Topological Sorting. Then $A$ is totally ordered.
    
    \textbf{Proof} Let $a_{1} a_{2}, \dots, a_{n}$ be the topological sorting of $A$. We will show that $a_{i} \le a_{i + 1}$. Suppose there is a value of $i$ for which $a_{i} \& a_{i + 1}$ are not comparable.
    
    Note that the topological sorting $(a_{1}, a_{2}, a_{3}, \dots, a_{i}, a_{i + 1}, \dots, a_{n}$
    
    There is also a topological sorting $a_{1}, a_{2}, \dots, a_{i + 1}, a_{i}, \dots, a_{n}$. Since we do not have a topological sorting of $A$, there are no two points that are not comparable.
\end{document}