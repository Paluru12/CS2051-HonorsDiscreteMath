\documentclass{article}
\usepackage[margin=1in]{geometry}
\usepackage{amsmath, amssymb, amsthm}
\usepackage{enumitem}

\usepackage{mdframed}
\usepackage{mathtools}
\usepackage{cancel}

%Formatting and Spacing
\setitemize[1]{noitemsep, parsep = 5pt, topsep = 5pt}
\setenumerate[1]{label = (\alph*), parsep = 1pt, topsep = 5pt}
\setlength\parindent{0pt}
\linespread{1.1}

%Cases Environment
\newlist{Cases}{enumerate}{3}
\setlist[Cases]{leftmargin = .25in, label = {Case \arabic*.}, topsep = 0.01in, itemsep = 0.04in, itemindent = .5in, parsep = 0in}

%Custom Title Fields
\newcommand{\lectTitle}{Lecture 16 Notes}
\newcommand{\lectTime}{March 14, 2022}
\newcommand{\lectClass}{Honors Discrete Mathematics}
\newcommand{\lectClassInstructor}{Gerandy Brito}
\newcommand{\lectSection}{Spring 2022}
\newcommand{\lectAuthorName}{Sarthak Mohanty}

%Headers and Footers
\usepackage{fancyhdr}
\usepackage{extramarks}
\pagestyle{fancy}
\lhead{\lectTime}
\chead{\lectClass \ (\lectClassInstructor)}
\rhead{\lectTitle}
\cfoot{\thepage}
\renewcommand\headrulewidth{0.4pt}
\renewcommand\footrulewidth{0.4pt}

\title{
    \vspace{2in}
    \textbf{\lectClass:\\ \lectTitle}\\
    \vspace{0.1in}\large{\textit{\lectClassInstructor\ \lectSection}}
    \vspace{3in}
    \author{\textbf{\lectAuthorName}}
    \date{}
}

\begin{document}

\maketitle
\pagebreak

\section*{Congruency Classes and Applications}

\subsection*{Invertibility}
    The extended Euclidean algorithm states that given $a \ge b > 0$, where $a, b \in \mathbb{N}$, we can construct $d$ such that $d = \gcd(a, b) = a \cdot r + b \cdot s$. This leads us to Bezout's Identity:
    
    \vspace{1.5mm}
    \textbf{Bezout's Identity.} If $d = \gcd(a, b)$, then $\exists r, s \in \mathbb{Z}$ such that $d = a \cdot r + b \cdot s$.
    
    \vspace{1.5mm}
    Let's use this identity to solve the system $ax \equiv 1 \pmod{m}$. Using B\'ezout's Theorem, we need that $\gcd{a, m} = 1$.
    \begin{align*}
        1 &= a \cdot s + m \cdot t \\
        1 &= a \cdot s \pmod{m},
    \end{align*}
    so $s$ is a solution.
    
    \vspace{1.5mm}
    \textbf{Definition.} Given $\Bar{a} \in \mathbb{Z}$, the \textit{inverse} of $\Bar{a}$ is $\Bar{b}$ such that $\Bar{a}\Bar{b} = \Bar{1}$.
    
    \textbf{Claim.} The inverse is unique
    

\subsection*{Chinese Remainder Theorem}
    As discussed last class, recall the Chinese Remainder Theorem, or CRT: 
    
    \vspace{1.5mm}
    \textbf{Definition.} Let $m_{1}, m_{2}, \dots, m_{k} \in \mathbb{N}^{*}$ such that $\gcd(m_{i}, m_{j}) = 1$ for all $1 \le i < j \le k$. Also let $a_{1}, a_{2}, \dots, a_{k} \in \mathbb{Z}$. The system
    $$\begin{cases}
        x &\equiv a_{1} \pmod{m_{1}} \\
        x &\equiv a_{2} \pmod{m_{2}} \\
        &\vdotswithin{ = } \\
        x &\equiv a_{k} \pmod{m_{k}}.
    \end{cases}$$
    has a unique solution $\mod M = m_{1}m_{2}\dots m_{k}$.
    
    \vspace{1.5mm}
    \textit{Proof of Existence. } Let $$M_{i} = \frac{M}{m_{i}} = m_{1} \cdot m_{2} \cdot m_{3} \cdot \cdots \cdot m_{i - 1} \cdot m_{i + 1} \cdot m_{k}.$$ such that $\gcd(m_{i}, M_{i}) = 1$.
    Then we have $$x = \cancelto{0}{a_{1}y_{1}M_{1}} + \dots + \cancelto{0}{a_{i - 1}y_{i - 1}M_{i - 1}} + a_{i}y_{i}M_{i} + \cancelto{0}{a_{i + 1}y_{i + 1}M_{i + 1}} + \cdots + \cancelto{0}{a_{k}y_{k}M_{k}} \equiv \cancelto{1}{y_{i}M_{i}}a_{i} \equiv a_{i} \pmod{m_{i}}$$ where $y_{i} \cdot M_{i} \equiv 1 \pmod{m_{i}}$; in other words, $y_{i}$ is the inverse of $M_{i}$ mod $m_{i}$.
    
    \vspace{1.5mm}
    \textit{Proof of ``Uniqueness". } Let $x, \hat{x}$ be two solutions. Then $$x_{1} \equiv x_{2} \equiv a_{i} \quad (\forall i)(1 \le i \le k).$$ Then by definition $x_{1} - x_{2}$ is a multiple of $m_{i}$. But then $M = m_{1}m_{2}\dots m_{k}$ divides $x_{1} - x_{2}$. By definition, this means $x_{1} \equiv x_{2} \pmod{M}$. If $x_{1}, x_{2} < M$, this means $x_{1} = x_{2}$.

\subsection*{Examples}
    1. P1 (IMO 1959) Show that $\frac{21n + 4}{14n + 3}$ is irreducible for all $n \in \mathbb{N}$.
    
    \vspace{1.5mm}
    \textit{Proof. } Recall $\gcd(a, b) = \gcd(b, a - b)$. Hence 
    $$\begin{aligned}[t]
        \gcd(21n + 4, 14n + 3) &= \gcd(14n + 3, 7n + 1)\\
        &= \gcd(7n + 1, 7n + 2) \\
        &= \gcd(7n + 1, 1) = 1
    \end{aligned}$$
    
    
    2. Consider the sequence $\{a_{n}\}_{n \in \mathbb{N}}$, where $a_{n} = 100 + n^{2}$. Let $d_{n} = \gcd(a_{n}, a_{n + 1}$. find the largest value of $d$.
    
    \vspace{1.5mm} Solution. In other words, we are looking for
    $$\begin{aligned}[t]
        \gcd(100 + n^{2}, 100 + n^{2} + 2n + 1) &= \gcd(100 + n^{2}, 2n + 1)
    \end{aligned}$$
    Note that $2n + 1$ is always odd. We claim that $d$ is always odd, since the divisor of an odd number must be odd. We now make the following claim:
    
    \vspace{1.5mm}
    If $d = \gcd(a, b)$ is odd and $b$ is odd, then $d = \gcd(2^{k}a, b)$, where 
    
    We can now simplify
    $$\begin{aligned}[t]
        \gcd(100 + n^{2}, 2n + 1) &= \gcd(400 + 4n^{2}, 2n + 1) \\
        &= \gcd((2n + 1)^{2} - 4n - 1 + 400, 2n + 1) \\
        &= \gcd((2n + 1)^{2} - 4n + 399, 2n + 1) \\
        &= \gcd(-4n + 399, 2n + 1) \\
        &= \gcd(-4n + 399, 4n + 2) \\
        &= \gcd(401, 4n + 2)
    \end{aligned}$$
    Check for $n = 200$.

\section*{Post-Lecture}

\subsection*{Question 1}
    Show that $4^{1536} - 9^{4824}$ is a multiple of $35$.

\subsection*{Solution}
    By definition, it suffices to check that $9^{4824} \equiv 4^{1536} \pmod{35}$. Using the Chinese Remainder Theorem, both of the following equivalencies hold:
    \begin{align*}
        9^{4824} &\equiv 4^{1536} \pmod{5} \\
        9^{4824} &\equiv 4^{1536} \pmod{7}.
    \end{align*}
    We now use the rule that if $a \equiv b \pmod{m}$, then $a^{k} \equiv b^{k} \pmod{m}$.
    \begin{align*}
        9^{4824} &\equiv (-1)^{4824} = 1 \pmod{5} \\
        4^{1536} &\equiv (-1)^{1536} = 1 \pmod{5}.
    \end{align*}
    Thus $9^{4824} \equiv 4^{1536} \equiv 1 \pmod{5}$.
    
    Now 
    \begin{align*}
        9^{4824} &\equiv 2^{4824} \pmod{7} = 2^{3 \cdot 1608} = (8)^{1608} \equiv 1^{1608} = 1 \pmod{7} \\
        4^{1536} &= 4^{3 \cdot 512} = 64^{512} \equiv 1^{512} \equiv \pmod{7}.
    \end{align*}
    Thus $9^{4824} \equiv 4^{1536} \equiv 1 \pmod{7}$, and by CRT we have our desired result.

\end{document}