\documentclass{article}
\usepackage[margin=1in]{geometry}
\usepackage{amsmath, amssymb, amsthm}
\usepackage{enumitem}

%Highlighting
\usepackage{xcolor, soul}
\sethlcolor{lightgray}

%Cases Environment
\newlist{Cases}{enumerate}{3}
\setlist[Cases]{leftmargin = .25in, label = {Case \arabic*.}, topsep = 0.01in, itemsep = 0.04in, itemindent = .5in, parsep = 0in}

%Formatting and Spacing
\setitemize[1]{noitemsep, parsep = 5pt, topsep = 5pt}
\setenumerate[1]{label = (\alph*), parsep = 1pt, topsep = 5pt}
\setlength\parindent{0pt}
\linespread{1.15}

%Custom Title Fields
\newcommand{\lectTitle}{Lecture 7 Notes}
\newcommand{\lectTime}{February 2, 2022}
\newcommand{\lectClass}{Honors Discrete Mathematics}
\newcommand{\lectClassInstructor}{Professor Gerandy Brito}
\newcommand{\lectSection}{Spring 2022}
\newcommand{\lectAuthorName}{Sarthak Mohanty}

%Headers and Footers
\usepackage{fancyhdr}
\usepackage{extramarks}
\pagestyle{fancy}
\lhead{\lectTime}
\chead{\lectClass \ (\lectClassInstructor)}
\rhead{\lectTitle}
\cfoot{\thepage}
\renewcommand\headrulewidth{0.4pt}
\renewcommand\footrulewidth{0.4pt}

\title{
    \vspace{2in}
    \textbf{\lectClass:\\ \lectTitle}\\
    \vspace{0.1in}\large{\textit{\lectClassInstructor\ \lectSection}}
    \vspace{3in}
    \author{\textbf{\lectAuthorName}}
    \date{}
}

\begin{document}

\maketitle
\pagebreak

\section*{Cardinality}
    
    There is an easy way to test if two numbers have the same cardinality, which will become useful when we deal with infinite sets.
    
    \vspace{1.5mm}
    \textbf{Theorem.} Two sets $A$ and $B$ are said to have the same cardinality if there exists a bijection (surjection?) from $A$ to $B$. The proof of this theorem is outside the scope of this course.
    
    \vspace{1.5mm}
    Now consider the collection of all sets (i.e.: the power set of the universe!) and define the relation
    $$A \mathcal{R }B \text{ iff $A$ and $B$ have the same cardinality.}$$
    
    Using the above theorem, we will now show that this relation is an equivalence relation.
    
    \begin{enumerate}[label = \arabic*.]
        \item Reflexivity: There exists a bijection from $A$ to $B$ $f: A \rightarrow B$.
        \item Symmetry: Since $a\mathcal{R}b$, we have defined a bijection from $A$ to $B$. As discussed before, bijections have inverses, so we can define an inverse $f^{-1} B \rightarrow B$, so $b \mathcal{R} a$.
        \item Transitivity: By definition, there exists a bijection from $A$ to $B$. Furthermore, there exists a bijection from $B$ to $C$. The composition $h: A \rightarrow C = g \circ f$ is a bijection as well (proof shown below), so $a \mathcal{R} c$.
    \end{enumerate}
    
    \vspace{1.5mm}
    \textbf{Proposition.}
    Let $f: A \rightarrow B$ and $g: B \rightarrow C$ be bijections. Then $h: A \rightarrow C$ is a bijection as well.
    
    \textbf{Proof.}
    Let’s start by proving that $g \circ f$ is one-to-one. Suppose $g \circ f(a) = g \circ f(b)$. Then $g(f(a)) = g(f(b))$. Since $g$ is one-to-one, this implies $f(a) = f(b)$. Since $f$ is one-to-one, $a = b$. Therefore $g \circ f$ is one-to-one. 
    \vspace{1.5mm}
    
    \qquad Let’s now prove that $g \circ f$ is onto. Suppose $c \in C$. Since $c \in C$ and $g$ is onto, there exists $b \in B$ such that $g(b) = c$. Since $b \in B$ and f is onto, there exists an $a \in A$ such that $f(a) = b$. Therefore $g(f(a)) = c$, i.e., $g \circ f(a) = c$. Since $g \circ f$ is one-to-one and onto, it is a bijection.
    
    \vspace{1.5mm}
    \textbf{Definition}: A set is said to be \textit{countable} if it is finite or has the same cardinality as the natural numbers.

    Examples: 
    Show the following are countable sets by defining bijections for each of the following sets.
    
    \begin{itemize}
        \item From $\mathbb{N}$ to the set of positive even numbers.
        \item From $\mathbb{N}$ to the set of negative integers.
        \item From $\mathbb{N}$ to the set of odd positive numbers.
        \item 
    \end{itemize}
    
    Solution:
    \begin{itemize}
        \item $f(n) = 2n$
        \item $f(n) = -n$
        \item  Let $A$ denote the set of odd natural numbers. Let $f(x) = 2x - 1$ for all $x \in \mathbb{N}$. Then $f: \mathbb{N} \rightarrow A$ is a bijection.
        \item 
    \end{itemize}
    
    
    \vspace{1.5mm}
    \textbf{Lemma.} Let $A, B$ be countable sets. Then $A \cup B$ is also countable.
    
    \vspace{1.5mm}
    \textbf{Proof.}
    Suppose both $|A|$ and $|B|$ is finite. Then the proof is trivial.
    
    Now suppose at least one of $|A|$, $|B|$ is infinite. WLOG, suppose $|A$ is infinite and the size of $b$ is $k$. Define a series of elements with the element at index $i$ equal to the $i$-th element of $|A$. Then shift every element $k$ spaces to the right, and fill the first $k$ spaces in the series with the elements in $B$. Since series are functions, we have shown there exists a bijection from $|A|$ to $|B|$
    
    Now suppose both $|A|$ and $|B|$ are infinite. Let $f(n) = 2n$ and $g(n) = 2n - 1$
    Define a function $h(n)$ as 
    $$h(n) = 
    \begin{cases}
        f(n) \text{ if $n$ is even}
        g(N) \text{ if $n$ is odd}
    \end{cases}$$
    
    We now have another tool to show set are countable: The finite union of countable sets is also countable.

    \vspace{1.5mm}
    \textbf{Lemma.} Let $A, B$ be countable sets. If $A \subseteq B$ and $B$ is countable, then so is $A$. Proof is left as an exercise for the reader.
    
    \vspace{1.5mm}
    \textbf{Lemma.} Let $A, B$ be countable sets. Then $A \times B$ is countable.
    
    \vspace{1.5mm}
    \textbf{Proof.} (TBC)
    We know $A \times B = \{a_{i}, b_{j} : a_{i} \in A, b_{j} = B\}$.
    
    Define a function $f$ as follows:
    \begin{align*}
        &f(a_{1}, b_{1}) = 1, \\
        &f(a_{2}, b_{a}) = 2, \quad f(a_{1}, b_{2}) = 3, \\
        &f(a_{3}, b_{1}) = 4, \quad f(a_{3}, b_{2}) = 5, \quad f(a_{3}, b_{3}) = 6, \\
        &f(1,4) = 7, \quad f(2, 3) = 8, \quad f(3, 2) = 9, \quad f(4, 1) = 10,
    \end{align*}
    
    and so on. Then $f: \mathbb{A} \times \mathbb{B} \rightarrow \mathbb{N}$ is a bijection.
    
    
    \textbf{Example.} 
    Show that $\mathbb{N}$ to $A = \{x \in \mathbb{Q}: x > 0\}$ is a bijection.
    
    We have many options, we can use the previous lemma and define the Cartesian product as $A \times B$, where $A$ is the set of all numerators and $B$ is the set of all denominators.
    
    
    Let $A$ denote the set of positive rational numbers. Define a function $f$ as follows:
    \begin{align*}
        & f(1) = \frac{1}{1}, \\
        & f(2) = \frac{1}{2}, \quad f(3) = \frac{2}{1}, \\
        & f(4) = \frac{1}{3}, \quad f(5) = \frac{3}{1}, \\
        & f(6) = \frac{1}{4}, \quad f(7) = \frac{2}{3}, \quad f(8) = \frac{3}{2}, \quad f(9) = \frac{4}{1},
    \end{align*}

\subsection*{Cantor's Diagonal Lemma}
    Let $f$ be a function from $\mathbb{N}$ to $(0, 1)$. Prove that there exists $y \in (0, 1)$ such that $y$ does not belong to the range of $f$. (in other words, prove the set of real numbers is not countable.)

\subsection*{Solution}
    We are given a function $f: \mathbb{N} \rightarrow (0, 1)$. We wish to find a number $y \in (0, 1)$ such that $$y \notin \{\text{$f(1)$, $f(2)$, $f(3)$, $f(4)$, \dots}\}.$$ For each $n \in \mathbb{N}$ and each $k \in \mathbb{N}$, let $x_{nk}$ be the $k$-th digit in the \textit{standard} decimal expansion of $f(n)$. Then 
    \begin{align*}
        &f(1) = 0.\text{\hl{$x_{11}$}}x_{12}x_{13}x_{14}\dots, \\
        &f(2) = 0.x_{21}\text{\hl{$x_{22}$}}x_{23}x_{24}\dots, \\
        &f(3) = 0.x_{31}x_{32}\text{\hl{$x_{33}$}}x_{34}\dots, \\
        &f(4) = 0.x_{41}x_{42}x_{43}\text{\hl{$x_{44}$}}\dots, \\
        &\text{and so on}.
    \end{align*}
    We shall define the number $y$ by defining the digits in its decimal expansion so that they are different from the ``diagonal" entries $x_{11}, x_{22}, x_{33}, x_{44}, \dots$ that are highlighted in the equations above. For each $n \in \mathbb{N}$, let
    $$y_n = 
    \begin{cases}
        5 & \text{if } x_{nn} \ne 5, \\
        4 & \text{if } x_{nn} = 5.
    \end{cases}$$
    Then for each $n \in \mathbb{N}$, $y_n \ne x_{nn}$. Now let $y$ be the number whose standard decimal expansion is $$y = 0.y_1y_2y_3y_4 \dots.$$ Then $y \in (0, 1)$. In fact, $0.444 \ldots \le y \le 0.555 \dots$. To see that $y$ is not in the range of $f$, note that for each $n \in \mathbb{N}$, $y \ne f(x)$ (because the numbers $y$ and $f(n)$ differ in their $n$-th decimal place; in other words, $y_n \ne x_{nn}$).

\section*{Post Lecture}
 
% \subsection*{Question 3}
%     Describe bijections (without justifications): Whenever the bijection is defined by a single formula, also provide its inverse.
%     \begin{enumerate}
%         \item from $\mathbb{Z}$ to $\mathbb{N}$.
%         \item \label{i} from $(-\frac{\pi}{2}, \frac{\pi}{2})$ to $\mathbb{R}$. [A suitable trigonometric function will do.]
%         \item from $(0, 1)$ to $\mathbb{R}$. [Compose a linear map with the map in part \ref{i}.]
%     \end{enumerate}

% \subsection*{Solution}
%     \begin{enumerate}
%         \item Let $n \in \mathbb{Z}$. Define $f(n)$ by
%         $$f(n) = 
%         \begin{cases}
%             2n + 1 & \text{if } n \ge 0, \\
%             -2n & \text{if } n < 0.
%         \end{cases}$$
%         Then $f: \mathbb{Z} \rightarrow \mathbb{N}$ is a bijection.
%         \item Let $f(x) = \tan(x)$ for all $x \in (-\frac{\pi}{2}, \frac{\pi}{2})$. Let $g(y) = \tan^{-1}(y)$ for all $y \in \mathbb{R}$. Then $f^{-1} = g$ and $f: (-\frac{\pi}{2}, \frac{\pi}{2}) \rightarrow \mathbb{R}$ is a bijection.
%         \item Let $f(x) = \tan(x)$ and $g(x) = \sin^{-1}(2x - 1)$. Now $f$ is a bijection from $(-\frac{\pi}{2}, \frac{\pi}{2})$ to $\mathbb{R}$ and $g$ is a bijection from $(0, 1)$ to $(-\frac{\pi}{2}, \frac{\pi}{2})$. Then $(f \circ g)(x) = \tan(\sin^{-1}(2x - 1))$ is a bijection from $(0, 1)$ to $\mathbb{R}$. The inverse of $f \circ g$ is $(g^{-1} \circ f^{-1})(y) = \frac{\sin(\tan^{-1}(y)) + 1}{2}$ for all $y \in \mathbb{R}$.
%     \end{enumerate}
    
    \subsection{Problem}
         The Grand Hotel has countable many rooms, numbered 1, 2, 3, \dots, and it is fully occupied.
    \begin{enumerate}
        \item Suppose that $k$ guests arrive at the hotel. Show how you can accommodate all new guests without removing any of the current guests.
    \begin{mdframed}
    Shift all existing guests $k$ rooms over, and place the $k$ guests in the rooms $1, \dots k$.
    \end{mdframed}
        \item Suppose that a countable infinite number of guests arrive. Show how you can accommodate all new guests without removing any of the current guests.
    \begin{mdframed}
    Move all existing guests in room $n$ to room $2n + 1$. This frees all rooms that can be expressed as $2n$ for the countably infinite number of guests to occupy.
    \end{mdframed}
        \item Suppose Hilbert (the owner of the Grand Hotel) expands the property to a second building, also with infinite rooms numbered $1, 2, 3, \dots$. Show how you can spread the current guests to fill every room in both buildings.
    \begin{mdframed}
    Take every guest with an odd room number $2k - 1$, where $k \in \mathbb{N}$ and move them into room number $k$ in the second building. \\
        Now take the remaining guests in the original building. Since their room numbers are even, we can represent their room number as $m = 2l$, where $l \in \mathbb{N}$. Move these guests into room number $l$ in the first building.
    \end{mdframed}
    \end{enumerate}
\end{document}