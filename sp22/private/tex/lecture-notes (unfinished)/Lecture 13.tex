\documentclass{article}
\usepackage[margin=1in]{geometry}
\usepackage{amsmath, amssymb, amsthm}
\usepackage{enumitem}

%Vertical Dots in Align Environment
\usepackage{mathtools}

%Cases Environment
% \newlist{Cases}{enumerate}{3}
% \setlist[Cases]{leftmargin = .25in, label = {Case \arabic*.}, topsep = 0.01in, itemsep = 0.04in, itemindent = .5in, parsep = 0in}

%Formatting and Spacing
\setitemize[1]{noitemsep, parsep = 5pt, topsep = 5pt}
\setenumerate[1]{label = (\alph*), parsep = 1pt, topsep = 5pt}
\setlength\parindent{0pt}
\linespread{1.1}

%Custom Title Fields
\newcommand{\lectTitle}{Lecture 11 Notes}
\newcommand{\lectTime}{February 21, 2022}
\newcommand{\lectClass}{Honors Discrete Mathematics}
\newcommand{\lectClassInstructor}{Gerandy Brito}
\newcommand{\lectSection}{Spring 2022}
\newcommand{\lectAuthorName}{Sarthak Mohanty}

%Headers and Footers
\usepackage{fancyhdr}
\usepackage{extramarks}
\pagestyle{fancy}
\lhead{\lectTime}
\chead{\lectClass \ (\lectClassInstructor)}
\rhead{\lectTitle}
\cfoot{\thepage}
\renewcommand\headrulewidth{0.4pt}
\renewcommand\footrulewidth{0.4pt}

\title{
    \vspace{2in}
    \textbf{\lectClass:\\ \lectTitle}\\
    \vspace{0.1in}\large{\textit{\lectClassInstructor\ \lectSection}}
    \vspace{3in}
    \author{\textbf{\lectAuthorName}}
    \date{}
}

\begin{document}

\maketitle
\pagebreak

\section*{Algorithmic Analysis}

\subsection*{Growth of Functions}
    d

\end{document}






\subsection*{\Exercise \ (idea)}
    Let $m, a_1, b_1, a_2, b_2 \in \mathbb{Z}$. Suppose that $a_1 \equiv b_1 \pmod m$ and $a_2 \equiv b_2 \pmod m$.
    \begin{enumerate}
        \item Prove that $a_1 + a_2 \equiv b_1 + b_2 \pmod m$.
        \item Prove that $a_1a_2 \equiv b_1b_2 \pmod m$.
        \item Prove that $a^{k} \equiv b^{k} \pmod m$ for any $k \in \mathbb{N}$.
    \end{enumerate}

\subsection*{Solution}
    \begin{enumerate}
        \item Since $m \mid b_1 - a_1$ and $m \mid b_2 - a_2$, we know $m \mid (b_1 - a_1) + (b_2 - b_1)$; rearranging, we find that $m \mid (b_1 + b_2) - (a_1 + a_2)$, so $b_1 + b_2 \equiv a_1 + a_2 \pmod m$.
        \item Since $m \mid b_1 - a_1$, it follows that $m \mid b_2(b_1 - a_1)$. Since $m \mid b_2 - a_2$, it follows that $m \mid a_1(b_2 - a_2)$. Then $m \mid [b_2(b_1 - a_1) + a_1(b_2 - a_2)]$; simplifying, we find that $m \mid b_1b_2 - a_1a_2$, so $a_1a_2 \equiv b_1b_2 \pmod{m}$.
        \item Let $P(n)$ be the sentence $$a \equiv b \pmod m \Rightarrow a^n \equiv b^n \pmod m.$$
        \textsc{Base Case}: $P(1)$ is true, since $a \equiv b \pmod m \Rightarrow a^1 \equiv b^1 \pmod m$ is always true. \\
        \textsc{Inductive Step}: Now let $n \in \mathbb{N}$ such that $P(n)$ is true. Then since $a \equiv b \pmod m$ and $a^n \equiv b^n \pmod m$, we know $a(a^n) \equiv b(b^n) \pmod m$, or equivalently, $a^{n + 1} \equiv b^{n + 1} \pmod m$. Hence $P(n + 1)$ is true as well. \\
        \textsc{Conclusion}: We have proved by induction that for each $n \in \mathbb{N}$, $P(n)$ is true.
    \end{enumerate}