\documentclass{article}
\usepackage[margin=1in]{geometry}
\usepackage{amsmath, amssymb, amsthm}
\usepackage{enumitem}
\usepackage{mathtools}
\usepackage{cancel}
\usepackage{color}

%Formatting and Spacing
\setitemize[1]{noitemsep, parsep = 5pt, topsep = 5pt}
\setenumerate[1]{label = (\alph*), parsep = 1pt, topsep = 5pt}
\setlength\parindent{0pt}
\linespread{1.1}

%Cases Environment
\newlist{Cases}{enumerate}{3}
\setlist[Cases]{leftmargin = .25in, label = {Case \arabic*.}, topsep = 0.01in, itemsep = 0.04in, itemindent = .5in, parsep = 0in}

%Custom Title Fields
\newcommand{\lectTitle}{Lecture ?? Notes}
\newcommand{\lectTime}{March 7, 2022}
\newcommand{\lectClass}{Honors Discrete Mathematics}
\newcommand{\lectClassInstructor}{Gerandy Brito}
\newcommand{\lectSection}{Spring 2022}
\newcommand{\lectAuthorName}{Tillson Galloway}
\newtheorem{theorem}{Theorem}

%Headers and Footers
\usepackage{fancyhdr}
\usepackage{extramarks}
\pagestyle{fancy}
\lhead{\lectTime}
\chead{\lectClass \ (\lectClassInstructor)}
\rhead{\lectTitle}
\cfoot{\thepage}
\renewcommand\headrulewidth{0.4pt}
\renewcommand\footrulewidth{0.4pt}

\title{
    \vspace{2in}
    \textbf{\lectClass:\\ \lectTitle}\\
    \vspace{0.1in}\large{\textit{\lectClassInstructor\ \lectSection}}
    \vspace{3in}
    \author{\textbf{\lectAuthorName}}
    \date{}
}

\begin{document}

\maketitle
\pagebreak

\section*{The Pigeonhole Principle}
Today, we'll introduce a powerful tool relating to functions called the \textbf{Pigeonhole Principle}. It's called this because someone once said, "You can't fit $n$ pigeons into $n - 1$ holes without stuffing two pigeons into the same hole."

\begin{theorem}
Given $n$ objects and $n - 1$ boxes, at least one box must contain at least two objects.
\end{theorem}
\begin{proof}
Assume for contradiction that each box contains $k \leq 1$ objects. Then there are at most $n - 1$ objects, which contradicts that there are $n$ objects.
\end{proof}
This sounds simple, but surprisingly, this can be applied to many things. In the following problems, we'll identify pigeons and holes.



{\bf Example 0} Show that if there are 30 students in a class, then at least two have last names that begin with the same letter.
\begin{proof}
30 pigeons, 26 holes. 
\end{proof}


{\it This is most like a warming up example. You can comment on how simple is to detect the pigeons and the holes, and transition to more elaborated examples.}\\

{\bf Example 1:} Let $n\in \mathbb{N}$. Show that there is a multiple of $n$ such that all its digits are 0s and 1s.

{\it Set $a_k=111\dots 1$, with $k$ 1s, $1\leq k\leq n+1$. There are $n$ holes corresponding to the $n$ congruency classes, so two of these numbers will be in the same hole, i.e.: they are congruent (mod $n$). The difference has the desired properties ($nq_1 + r - nq_2 + r = n(q_1 - q_2)$).}\\


\textcolor{blue}{Show the computation on the board, I think it helps students to see how the difference yields a number with a bunch of 0s and a bunch of 1s.}\\

{\bf Example 2} Five points are chosen inside a square of side $2$. Show that there are at least two points at distance $\leq \sqrt{2}$.

{\it Split the square into four smaller squares of side 1. These are the holes! At least two points will land on the same square and the max distance between would be the diagonal, which is $\sqrt{2}$.}\\

\textcolor{blue}{I like to show how we can try to get the points ``as far as possible'' by placing one on each corner and then the last one yields the answer. Then I remark how this is one configuration and not a proof! Then I go back and suggest splitting in four regions using the diagonals, but this also fails  since the max distance inside each triangle is 1. Feel free to use your own approach.}

\begin{theorem}
If $S, T$ are sets and there exists an injective function $f: S \mapsto T$, then $|S| \leq |T|$. Contrapositively, if $|S| \geq |T|$, then there is no one-to-one mapping from S \mapsto T.
\end{theorem}


\underline{Generalized Pigeonhole Principle}.

\textcolor{blue}{Write it down and go over how it is an inmediate extention of the previous. If you feel like, you can announce Theorem 1.}

\begin{theorem}
Given $n$ objects and $k$ boxes, at least one box must contain at least $\lceil \frac{N}{k} \rceil$ objects.
\end{theorem}
\begin{proof}
Assume for contradiction that every box contains strictly less than $\lceil \frac{N}{k} \rceil$ objects. Then each of the $n$ boxes contains at most $\lceil \frac{N}{k} \rceil - 1$. By the product rule, this means that the total number of objects is at most $k(\lceil \frac{N}{k} \rceil - 1) < k(\frac{N}{k} + 1 - 1) = N$, which is a contradiction.
\end{proof}

\textbf{A common type of problem asks for the minimum number of objects such that at least r of these objects must be in one of k boxes when these objects are distributed among the boxes.}
Strategy: pick each object one-by-one and find the minimum in the worst case. To find the worst case, try to think of how you can avoid putting each pick into a certain box.
\\

{\bf Example 3} Given 100 people, what is the min number of people that could have been born in the same month?  \\
{\it $\lceil 100/12 \rceil$ = 9 people.}


{\bf Example 4} In a meeting of six people, each pair is either friends or enemies. Show that there is a group of three people such that they are either friends or they enemies .  \\
{ \it Draw a complete graph with the 6 nodes. Then analyze a single node and the nodes it is connected to. By pigeonhole principle, at least 3 edges have the same color. Assume WLOG that 1 is connected to 2, 3, 4 and they are red. If any 23, 24, or 34 are red, the whole triangle is red. Then we can assume they are all blue. But now, 234 is blue. Therefore, a same colored triangle must exist.}




{\it This one is on the book and you can follow that solution.}\\

\textcolor{blue}{Now explain to them how we can use the Pigeonhole Principle backwards, to guarantee certain property with the minimum number of individuals.}

{\bf Example 5} a) How many cards must be selected from a standard deck of 52 cards to guarantee that at least three cards of the same suit are chosen?
b) How many must be selected to guarantee that at least three hearts are selected?

{\it Set up card configuration. $N$ cards selected one at a time, each is put in one of four boxes (each suit has a box). Using GPP, if N cards, then at least one box contains $\lceil \frac{N}{4} \rceil$ cards. If $\lceil \frac{N}{4} \rceil \geq 3$, then we have 3 cards of the same suit. $n/4 = 3 \implies 4*2 + 1 = 9$. After choosing the 8th card, there's no way to not have a 3rd card of some type.}\\


{\it HEARTS: Not a GPP question. In the worst case, we choose all but hearts first then 3 hearts. 13 * 3 + 3 = 42nd card is the hearts.}\\

\textcolor{blue}{If time allows, you can do some clever proofs:}

{\bf Example 6} Show that every sequence of $n^2+1$ distinct real numbers contains a subsequence of length $n + 1$ that is either strictly increasing or strictly decreasing.
$a_1, a_2, \cdots, a_{n^2+1}$ in a sequence. $(i_k, d_k)$ are the longest subsequences starting at each index. Assume for contradiction that there all are $\leq n$. Then there are $n^2$ ways to count the pair, and by PP, there must be a pair with the same count. Assume the ones with the same count are $i_s, i_t$ where $s < t$. Two cases: if $a_s < a_t$, then you can create a new subsequence with $a_s$ and the subsequence starting at $a_t$, and the longest subsequence is now $i_t + 1$. If $a_t > a_s$, similar argument works for decreasing.


{\it This is given as a Theorem on the book, but the proof is very nice. I am a visual person and I would do the sequence on the board, mark $a_s$ and $a_t$ (this notation is from the proof in the book) and will then convey the final idea, but you are free to choose how to cover it}.\\

\textcolor{blue}{This should be sufficient, but if you need more, there is a section on the book with ``Elegant applications'' and you can use the baseball team example, or you can bring your own.}



\section*{Post-Lecture}
TBD

\end{document}