\documentclass{article}
\usepackage[margin=1in]{geometry}
\usepackage{amsmath, amssymb, amsthm}
\usepackage{enumitem}
\usepackage{mathtools}
\usepackage{cancel}

%Formatting and Spacing
\setitemize[1]{noitemsep, parsep = 5pt, topsep = 5pt}
\setenumerate[1]{label = (\alph*), parsep = 1pt, topsep = 5pt}
\setlength\parindent{0pt}
\linespread{1.1}

%Cases Environment
\newlist{Cases}{enumerate}{3}
\setlist[Cases]{leftmargin = .25in, label = {Case \arabic*.}, topsep = 0.01in, itemsep = 0.04in, itemindent = .5in, parsep = 0in}

%Custom Title Fields
\newcommand{\lectTitle}{Lecture 14 Notes}
\newcommand{\lectTime}{March 7, 2022}
\newcommand{\lectClass}{Honors Discrete Mathematics}
\newcommand{\lectClassInstructor}{Gerandy Brito}
\newcommand{\lectSection}{Spring 2022}
\newcommand{\lectAuthorName}{Sarthak Mohanty}

%Headers and Footers
\usepackage{fancyhdr}
\usepackage{extramarks}
\pagestyle{fancy}
\lhead{\lectTime}
\chead{\lectClass \ (\lectClassInstructor)}
\rhead{\lectTitle}
\cfoot{\thepage}
\renewcommand\headrulewidth{0.4pt}
\renewcommand\footrulewidth{0.4pt}

\title{
    \vspace{2in}
    \textbf{\lectClass:\\ \lectTitle}\\
    \vspace{0.1in}\large{\textit{\lectClassInstructor\ \lectSection}}
    \vspace{3in}
    \author{\textbf{\lectAuthorName}}
    \date{}
}

\begin{document}

\maketitle
\pagebreak

\section*{Congruency Classes}

\section*{Chinese Remainder Theorem}

\textbf{Example. } Find $x$ such that
    \begin{align*}
        x \equiv 1 \pmod{3} \\
        x \equiv 4 \pmod{5} \\
        x \equiv 6 \pmod{7}
    \end{align*}

\textbf{Solution 1. }

Recall: The unique solution $\pmod{3 \times 5 \times 7}$ is $$a_{1}M_{1}y_{1} + a_{2}M_{2}y_{2} + a_{3}M_{3}y_{3} \pmod{105}$$ where $a_{1} = 1$, $a_{2} = 4$, $a_{3} = 6$, $M_{1} = 35 = 5 \times 7$, $M_{2} = 21 = 3 \times 7$, and $M_{3} = 15 = 3 \times 5$. We wish to find $y_{i}$, $1 \le i \le 3$. Using the usual method for finding the inverse of a number, we find that $y_{1} = 2$, $y_{2} = 1$, and $y_{3} = 1$. Plugging into our formula for $x$, we find that $x \equiv 34 \pmod{105}$.

\subsection*{Euler's Theorem}
    (A generalization of FLT) Let $\gcd(a, N) = 1$. Then $a^{\phi(N)} \equiv 1 \pmod{N}$ where $\phi(N)$ is Euler's Totient Function: $\phi(N) = \{x : 1 \le x < N, \gcd(x, N) = 1\}$; in other words, the number of positive integers upto $N$ that are relatively prime with $N$. Here are a few facts about this function:
    \begin{itemize}
       \item $1 \le \phi(N) < N$
       \item $\phi(p) = p - 1$ for $p$ prime (this is how we can prove FLT from Euler's Theorem.
       \item $\phi(ab) = \phi(a) \cdot \phi(b) \forall a, b, \in \mathbb{N}$ (need $\gcd(a, b) = 1?$)
       \item $\phi(p^{k}) = p^{k - 1}(p - 1)$ where $p$ is prime $k \in \mathbb{N}$.
       
       Proof: Count all the way from $1$ to $p^{k}$, then to get the number of non-prime relative numbers by counting counter examples who are less than $p^{k}$ we know these are some of them. $p, \dots, 2p, \dots, 3p, \dots, (p^{k} - p)$ this is $p^{k - 1} - 1$ examples, futher more they are the only ones because of the prime decomposition of $p^{k}$. Thus: $p^{k} - 1 - (p^{k -1} - 1) = p^{k - 1}(p - 1)$. QED
       \item In particular, if $p$ and $q$ are primes, $\phi(pq) = (p - 1)(q - 1)$ with the above results you can decompose all numbers. Now we just need to figure out if a number if prime or not (This is usually quite tricky in practice.)
    \end{itemize}

\subsection*{Finding Primes}
    We have defined prime numbers using the following proposition: a number $n$ is prime iff $$(\forall a, b \in \mathbb{N})(n = ab \implies (a = 1 \lor b = 1)).$$
    
    We can use FLT to (usually) find prime numbers using the following procedure.
    \begin{itemize}
        \item Input: some $n \in \mathbb{N}$.
        \item Pick $a$ from $\{a_{1}, a_{2}, \dots, a_{[phi(N)}\}$ where $a_{i}$ are numbers that share no common factors with $N$.
        \item If $a^{N - 1} \equiv 1 \pmod{N}$ output Yes, otherwise no.
    \end{itemize}
    However, this procedure is not perfect. We define a new class of numbers: Carmichael numbers, as numbers that
    \begin{itemize}
        \item Are not prime
        \item $(\forall a \in \mathbb{N})(\gcd(a, N) = 1 \implies a^{N - 1} \equiv 1 \pmod{N}$
    \end{itemize}
    There are not many Carmichael numbers which are not primes, thus Fermat's results can be used somewhat effectively to find primes.
    The following table summarizes the results from this section.
    \begin{tabular}{c|c}
        $n$ is prime & Fermat's Test \\
        Yes & Yes \\
        Now & Probably no (not always)
    \end{tabular}


\section*{Post-Lecture}

\subsection*{Question 1}
    Show that $n \mid 2^{n!} - 1$ for all odd positive $n$.

\subsection*{Solution}
    $n=\prod\limits_{i=1}^rp_i^{e_i} \implies \;\varphi(n)=\prod_{i=1}^rp_i^{e_i-1}(p_i - 1)\; $ is clearly a divisor of $n!$ since each of its factors is less than $n$ and they're all distinct, so by Euler's Theorem,
    
    $$2^{n!}= \bigl(2^{\varphi(n)} \bigr)^{\tfrac{n!}{\varphi(n)}}\equiv 1\pmod{n}$$


\end{document}