\documentclass{article}
\usepackage[margin=1in]{geometry}
\usepackage{amsmath, amssymb, amsthm}
\usepackage{enumitem}

%Formatting and Spacing
\setitemize[1]{noitemsep, parsep = 5pt, topsep = 5pt}
\setenumerate[1]{label = (\alph*), parsep = 1pt, topsep = 5pt}
\setlength\parindent{0pt}
\linespread{1.15}

%Custom Title Fields
\newcommand{\lectTitle}{Lecture 2 Notes}
\newcommand{\lectTime}{January 12, 2022}
\newcommand{\lectClass}{Honors Discrete Mathematics}
\newcommand{\lectClassInstructor}{Professor Gerandy Brito}
\newcommand{\lectSection}{Spring 2022}
\newcommand{\lectAuthorName}{Sarthak Mohanty}

%Headers and Footers
\usepackage{fancyhdr}
\usepackage{extramarks}
\pagestyle{fancy}
\lhead{\lectTime}
\chead{\lectClass \ (\lectClassInstructor)}
\rhead{\lectTitle}
\cfoot{\thepage}
\renewcommand\headrulewidth{0.4pt}
\renewcommand\footrulewidth{0.4pt}

\title{
    \vspace{2in}
    \textbf{\lectClass:\\ \lectTitle}\\
    \vspace{0.1in}\large{\textit{\lectClassInstructor\ \lectSection}}
    \vspace{3in}
    \author{\textbf{\lectAuthorName}}
    \date{}
}

\begin{document}

\maketitle
\pagebreak

\section*{Predicates \& Quantifiers}
    \textbf{Definition:} A propositional function, or \textit{predicate}, can be thought of as a function taking in one or more variables, and outputting either True or False.
    \begin{itemize}
        \item Predicates are not propositions on their own, but as soon as we assign a value to our variables, they become a proposition.
        \item Predicates are usually denoted using the symbols $P(x), Q(x), \dots$, but this notation is not too rigid.
    \end{itemize}
    
    As predicates are dependent upon variables, we must introduce domains upon these variables. We do so using quantifiers.
    
    \vspace{1.5mm}
    \textbf{Definition: }A \textit{quantifier} is an expression that indicates the scope or domain to which it is attached.
        \begin{itemize}
            \item The most common quantifiers are the terms `for all' and `there exists', which are expressed using the symbols $\forall$ and $\exists$ respectively.
        \end{itemize}
    
    We can now express propositions we introduced yesterday using our new logical notation. The statement ``For all real numbers $x$, $x + 1 > x$" is logically equivalent to $(\forall x \in \mathbb{R})(P(x))$. The statement ``There exists a pair of irrational numbers $x, y$ such that $x^{y}$ is rational" is logically equivalent to $(\exists x, y \in \mathbb{R} \setminus \mathbb{Q})(P(x, y))$.
    
    \vspace{1.5mm}
    Propositions of this form can also be negated. We have 
    \begin{align*}
        \neg(\forall x)(P(x)) \equiv (\neg\forall x)(\neg P(x)) \equiv (\exists x)(\neg P(x)) \\
        \neg(\exists x)(P(x)) \equiv (\neg\exists x)(\neg P(x)) \equiv (\forall x)(\neg P(x)).
    \end{align*}

\section*{Implications}
    In this section, we will return to the conditional statement introduced last lecture, also known as an implication. Let us first introduce new types propositional statements branching off from implications.
    \begin{itemize}[itemsep = 0.2mm, parsep = 5pt, topsep = 5pt]
        \item The \textit{converse} of an implication $p \Rightarrow q$ is the statement $q \Rightarrow p$.
        \item The \textit{inverse} of an implication $p \Rightarrow q$ is the statement $\neg p \Rightarrow \neg q$
        \item the \textit{contrapositive} of an implication $p \Rightarrow q$ is the statement $\neg q \Rightarrow \neg p$.
    \end{itemize}
    The truth table of these statements is shown below.
    $$\begin{array}{|c|c|c|c|c|c|}
        \hline
        p & q & p \Rightarrow q & q \Rightarrow p & \neg p \Rightarrow \neg q & \neg q \Rightarrow \neg p\\
        \hline
        T & T & T & T & T & T\\
        T & F & F & T & T & F \\
        F & T & T & F & F & T \\
        F & F & T & T & T & T \\
        \hline
    \end{array}$$
    Note that the contrapositive of an implication is logically equivalent to the implication itself. For this reason, we introduce a new type of proof, known as a \textit{Proof by Contraposition}.
    
    \subsection*{Proof by Contraposition}
    
    \vspace{1.5mm}
    \textbf{Example} \\
    Show that if $n^2$ is odd, then $n$ is odd.
    
    \vspace{1.5mm}
    \textbf{Solution} \\
    We will prove the contrapositive of this statement, in other words, ``if $n$ is even, then $n^{2}$ is even."
    
    \vspace{1.5mm}
    Suppose $n$ is even. Then there exists an integer $k$ such that $n = 2k$. Squaring both sides, we obtain $n^{2} = 4k^{2} = 2(2k^{2})$, and by definition $n^{2}$ is even. Hence the contrapositive statement is true, so the original statement is true as well.
        
\section*{Negating Compound Propositions}
    Now we will show how to negate disjunctions, conjunctions, and implications.

\subsection*{Negating Disjunctions}
    Suppose your roommate made the prediction ``I will get an A in CS 2051 and an A in CS 1332." If your roommate's prediction is incorrect, then they either did not get an A in CS 2051, or they did not get an A in CS 1332. This leads us to the first De Morgan's law: $$\neg(p \lor q) \equiv   \neg p \land \neg q$$

\subsection*{Negating Conjunctions}
    Suppose your roommate made the prediction ``I will get an A in CS 2051 or an A in CS 1332." If your roommate's prediction is incorrect, then they did not get an A in CS 2051 and did not get an A in CS 1332. This leads us to the second De Morgan's law: $$\neg (p \land q) \equiv \neg p \lor \neg q.$$
    
    \vspace{1.5mm}
    Note that De Morgan’s laws can be extended to expressions with more than one conjunction and disjunction as follows: 
    \begin{align*}
        \neg (p \land q \land r) &\equiv \neg p \lor \neg q \lor \neg r \\
        \neg (p \lor q \lor r) &\equiv \neg p \land \neg q \land \neg r
    \end{align*}
        
\subsection*{Negating Implications}
    One important theorem that will show up in your later CS courses is that any logical statement can be expressed using the basic operators $\lor$, $\land$, or $\neg$. Even implications can be expressed as such, leading us to the implication law: $$p \Rightarrow q = \neg p \lor q.$$ Using De Morgan's law, we can now negate this statement, leading to the implication negation law: $$\neg(p \Rightarrow q) = p \land q.$$
    
    \vspace{1.5mm}
    \textbf{Example}
    
    Consider the predicate ``If a natural number $x$ is greater than one, then $x$ is even." When would this statement be false?
    
    \vspace{1.5mm}
    \textbf{Solution}
    
    Let $P(x)$ represent the statement ``$x$ is greater than one" and let $Q(x)$ represent the statement ``$x$ is even". Then the original predicate is equivalent to $(x \in \mathbb{N})(P(x) \Rightarrow Q(x))$. Using the implication negation law, the negation of this predicate is $(x \in \mathbb{N})(P(x) \land \neg Q(x))$. Hence this statement is false when $x$ is greater than one and $x$ is odd.
        
\section*{Arguments}
    Arguments are a series of statements, known as premises, used to prove a conclusion.
    
    \vspace{1.5mm}
    Logically, this can be visualized as $(P_{1} \land P_{2} \land \dots \land P_{k}) \Rightarrow t$, where $P_{1}, P_{2}, \dots, P_{k}$ are propositions with a truth value of True, and $t$ is a proposition with an unknown truth value. See the textbook for basic examples of arguments.

\end{document}