\documentclass{article}
\usepackage[margin=1in]{geometry}
\usepackage{amsmath, amssymb, amsthm}
\usepackage{enumitem}

%Vertical Dots in Align Environment
\usepackage{mathtools}

%Cases Environment
% \newlist{Cases}{enumerate}{3}
% \setlist[Cases]{leftmargin = .25in, label = {Case \arabic*.}, topsep = 0.01in, itemsep = 0.04in, itemindent = .5in, parsep = 0in}

%Formatting and Spacing
\setitemize[1]{noitemsep, parsep = 5pt, topsep = 5pt}
\setenumerate[1]{label = (\alph*), parsep = 1pt, topsep = 5pt}
\setlength\parindent{0pt}
\linespread{1.1}

%Custom Title Fields
\newcommand{\lectTitle}{Lecture 14 Notes}
\newcommand{\lectTime}{March 7, 2022}
\newcommand{\lectClass}{Honors Discrete Mathematics}
\newcommand{\lectClassInstructor}{Gerandy Brito}
\newcommand{\lectSection}{Spring 2022}
\newcommand{\lectAuthorName}{Sarthak Mohanty}

%Headers and Footers
\usepackage{fancyhdr}
\usepackage{extramarks}
\pagestyle{fancy}
\lhead{\lectTime}
\chead{\lectClass \ (\lectClassInstructor)}
\rhead{\lectTitle}
\cfoot{\thepage}
\renewcommand\headrulewidth{0.4pt}
\renewcommand\footrulewidth{0.4pt}

\title{
    \vspace{2in}
    \textbf{\lectClass:\\ \lectTitle}\\
    \vspace{0.1in}\large{\textit{\lectClassInstructor\ \lectSection}}
    \vspace{3in}
    \author{\textbf{\lectAuthorName}}
    \date{}
}

\begin{document}

\maketitle
\pagebreak

\textbf{TA Remark.} It's only Monday and I'm already tired :(

\section*{Division}
    \textbf{Definition. } We said $d$ is the greatest common divisor of $a$ and $b$ (notation: $d = \gcd(a, b)$) if $d \mid a$, $d \mid b$, and $d$ is the largest integer with these properties.
    
    \vspace{1.5mm}
    \textbf{Lemma.} Let $a \ge b > 0$. Then $\gcd(a, b) = gcd(a - b, b)$.
    
    \vspace{1.5mm}
    \textit{Proof. } Set $d = \gcd(a, b)$ and $\hat{d} = \gcd(a - b, b)$
    \begin{enumerate}[label = \arabic*.]
        \item We will show first that $d \le \hat{d}$. $d \mid a$ and $d \mid b$ by definition of $d$ . Then $d \mid a - b$ and $d$ is a common divisor of $a - b$ and $b$. $d \le \hat{d}$ by definition of $\hat{d}$.
        \item Now we show that $\hat{d} \le d$. $\hat{d} \mid a - b$ and $\hat{d} \mid b$ by definition. Then $\hat{d} \mid ((a - b )+ b)$. $\hat{d}$ is also a common divisor of $a$ and $b$, implying $\hat{d \le d}$ by definition.
    \end{enumerate}
    Note the same proof leads to $\gcd(a, b) = \gcd(a - b, a)$.
    
    % We can use this result to devise an algorithm to find the greatest common divisor of two integers.
    
    $$\begin{aligned}[t]
        \gcd(a, b) &= \gcd(a - b, b) \\
        &= \gcd(a - 2b, b) \\
        &= \gcd(a - 3b, b) \\
        &\vdotswithin{ = } \\
        &= \gcd(a - qb, b)
    \end{aligned}$$
    
    \textbf{Reminder.} We said $d$ divides $a$ if $\exists q \in \mathbb{Z}$ such that $a = qd$. 
    
    \vspace{1.5mm}
    \textbf{Claim.} Let $a, d \in \mathbb{N}$. There exists a unique pair $(q, r) \in \mathbb{N}$ such that: 
    \begin{itemize}
        \item $0 \le r \le d - 1$
        \item $a = q \cdot d + r$.
    \end{itemize}
    
    \vspace{1.5mm}
    \textit{Proof of Claim. }
    \begin{itemize}
        \item Existence: Follows from long division.
        \item Uniqueness: Assume $\exists(q, r)$ and $\exists(q', r')$ such that $a = q \cdot d + r$ and $a = q'd + r'$ and $0 \le r, r' \le d - 1$. We have $a = q \cdot d + r = q' \cdot d + r'$. Rearranging, we get $d(q - q') = r' - r < d$ (More precisely: $|r - r'| < d$). Thus, equality only holds if $q - q' = 0$ OR $q = q'$ and $r - r' = 0$ OR $r = r'$.
    \end{itemize}
    We now devise a procedure for finding the greatest common divisor, known as the Euclidean algorithm.
    \begin{enumerate}[leftmargin=1.2in]
        \item [\textbf{Initialization.}] Set $a_{1} = a$ and $b_{1} = b$.
        \item [\textbf{Inductive Step.}] Set $a_{i} = b_{i - 1}$, and set $b_{i} = a - qb$. Repeat this step, incrementing $i$ in each iteration, until $i = n$ such that $b_{n}$ is set equal to $0$.
        \item [\textbf{Termination.}] Return $\gcd(a, b)$, given by $a_{n}$.
    \end{enumerate}
    
    \textbf{Definition.} Consider $\mathcal{R}_{m}$ to be the equivalence relation over $\mathbb{Z}$: $$\text{$a \mathcal{R}_{m} b$ if the remainder of $a$ when divided by $m$ equals remainder of $b$ when divided by $m$}.$$ With this equivalence relation, we can partition the elements in $\mathbb{Z}$ int he form $[0], [1], [2],\dots, [m-1]$.
    
    \vspace{1.5mm}
    \textbf{TA Remark. }. These equivalency classes are known as congruency classes.
    
    \vspace{1.5mm}
    \textbf{Definition. } $\mathbb{Z}_{m}$ is the set of equivalence classes of $\mathcal{R}_{m} (|\mathbb{Z}_{m}| = m)$.
    
    \vspace{1.5mm}
    Notation:
    \begin{itemize}
        \item Elements in $\mathbb{Z}_{m}$ are denoted by $\Bar{a}$.
        \item $r = \min \{x \ge 0 \text{ in each equivalence class}\}$ is the representative of the class.
        \item $a \mathcal{R}_{m}$ is denoted as $a \equiv b \pmod{m}$.
    \end{itemize}
    
    \vspace{1.5mm}
    \textbf{Examples. }
    $m = 4$. \\
    Then $\mathbb{Z}_{m} = \{\Bar{0}, \Bar{1}, \Bar{2}, \Bar{3}\}$ where $\Bar{0} = \{-12, -8, -4, 0, 4, 8, 12, \dots\}$ and $\Bar{1} = \{-11, -7, -3, 1, 5, 9, 12, 17, \dots\}$.
    
    Operations on $\mathbb{Z}_{m}$. (Used on homework)
    \begin{itemize}
        \item $\Bar{a} + \Bar{b} = \overline{a + b}$
        \item $\Bar{a} + \Bar{b} = \overline{a \cdot b}$
    \end{itemize}
    In $\mathbb{Z}_{4}$:
    \begin{itemize}
        \item $\Bar{1} + \Bar{2} = \Bar{3}$
        \item $\Bar{6} + \Bar{8} = \Bar{14} \sim \Bar{2} + \Bar{0} = \Bar{2}$
        \item $\Bar{1} + \Bar{3} = \Bar{0}$
        \item $\Bar{2} + \Bar{3} = \Bar{1}$.
    \end{itemize}

\subsection*{Post Lecture}

\subsection*{Question 1}
    Let $m, a_1, b_1, a_2, b_2 \in \mathbb{Z}$. Suppose that $a_1 \equiv b_1 \pmod m$ and $a_2 \equiv b_2 \pmod m$.
    \begin{enumerate}
        \item Prove that $a_1 + a_2 \equiv b_1 + b_2 \pmod m$.
        \item Prove that $a_1a_2 \equiv b_1b_2 \pmod m$.
        \item Prove that $a^{k} \equiv b^{k} \pmod m$ for any $k \in \mathbb{N}$.
    \end{enumerate}

\subsection*{Solution}
    \begin{enumerate}
        \item Since $m \mid b_1 - a_1$ and $m \mid b_2 - a_2$, we know $m \mid (b_1 - a_1) + (b_2 - b_1)$; rearranging, we find that $m \mid (b_1 + b_2) - (a_1 + a_2)$, so $b_1 + b_2 \equiv a_1 + a_2 \pmod m$.
        \item Since $m \mid b_1 - a_1$, it follows that $m \mid b_2(b_1 - a_1)$. Since $m \mid b_2 - a_2$, it follows that $m \mid a_1(b_2 - a_2)$. Then $m \mid [b_2(b_1 - a_1) + a_1(b_2 - a_2)]$; simplifying, we find that $m \mid b_1b_2 - a_1a_2$, so $a_1a_2 \equiv b_1b_2 \pmod{m}$.
        \item Let $P(n)$ be the sentence $$a \equiv b \pmod m \Rightarrow a^n \equiv b^n \pmod m.$$
        \textsc{Base Case}: $P(1)$ is true, since $a \equiv b \pmod m \Rightarrow a^1 \equiv b^1 \pmod m$ is always true. \\
        \textsc{Inductive Step}: Now let $n \in \mathbb{N}$ such that $P(n)$ is true. Then since $a \equiv b \pmod m$ and $a^n \equiv b^n \pmod m$, we know $a(a^n) \equiv b(b^n) \pmod m$, or equivalently, $a^{n + 1} \equiv b^{n + 1} \pmod m$. Hence $P(n + 1)$ is true as well. \\
        \textsc{Conclusion}: We have proved by induction that for each $n \in \mathbb{N}$, $P(n)$ is true.
    \end{enumerate}
    
\end{document}
