\documentclass{article}
\usepackage[margin=1in]{geometry}
\usepackage{amsmath, amssymb, amsthm}
\usepackage{enumitem}

%Formatting and Spacing
\setitemize[1]{noitemsep, parsep = 5pt, topsep = 5pt}
\setenumerate[1]{label = (\alph*), parsep = 1pt, topsep = 5pt}
\setlength\parindent{0pt}
\linespread{1.1}

%Custom Title Fields
\newcommand{\lectTitle}{Lecture 1 Notes}
\newcommand{\lectTime}{January 10, 2022}
\newcommand{\lectClass}{Honors Discrete Mathematics}
\newcommand{\lectClassInstructor}{Professor Gerandy Brito}
\newcommand{\lectSection}{Spring 2022}
\newcommand{\lectAuthorName}{Tillson Galloway \& Sarthak Mohanty}

%Headers and Footers
\usepackage{fancyhdr}
\usepackage{extramarks}
\pagestyle{fancy}
\lhead{\lectTime}
\chead{\lectClass \ (\lectClassInstructor)} 
\rhead{\lectTitle}
\cfoot{\thepage}
\renewcommand\headrulewidth{0.4pt}
\renewcommand\footrulewidth{0.4pt}

\title{
    \vspace{2in}
    \textbf{\lectClass:\\ \lectTitle}\\
    \vspace{0.1in}\large{\textit{\lectClassInstructor\ \lectSection}}
    \vspace{3in}
    \author{\textbf{\lectAuthorName}}
    \date{}
}

\begin{document}

\maketitle
\pagebreak

\section*{Logistics}

\subsection*{Lectures}
    We're livestreaming and recording lectures so you can attend virtually. Everyone should be on Microsoft Teams (email Brito if you are missing). Lectures are mandatory: you are responsible for the announcements and content from lecture. We'll also be releasing notes for each lecture (you are here!).

\subsection*{Grading}
    This information is all on Canvas. One non-trivial thing is that the final exam is optional: if you choose to take the final, it will replace your lowest midterm. The final will be cumulative, however, while the midterms are not. We'll also have some straightforward Participation quizzes on Canvas each week. The quizzes and exams will be conducted \textbf{in-person} unless you have an excuse communicated to the teaching staff.

\section*{What is CS 2051?}
    The main purpose of this class is to learn proofs! 
    We'll start by developing techniques to prove that things are true in the language of math, then we'll move onto some topics that use that language to talk about some more interesting things. This is the honors section, so we'll have some freedom to explore additional topics and will go more in-depth with the listed ones.
    

\section*{Logic}
    The goal of discrete mathematics is to translate our current model of information (English, as an example), into an universal language. We call this language Logic. The precise definition of ``logic" is quite broad. Let us first introduce propositional logic.

\section*{Propositions}
    To begin, we have the following definition.
    
    \vspace{1.5mm}
    \textbf{Definition} \textit{A proposition is a statement that can be always either true or false  (never both).}
    
    \vspace{1.5mm}
    A few examples:
    \begin{itemize}
        \item Professor Brito is Cuban. (True)
        \item 2 + 3 = GT. (False)
        \item ``If we go to the park, then we play baseball." Both the antecedent (the part after the `if') and the consequent (the part after the `then') are propositions, so the entire statement is True.
    \end{itemize}
    
    An example of a statement that is \textit{not} a proposition  is the statement ``I am inevitable", because there is no objective definition for what `inevitable' is. Let's conclude with one more definition.
    
    \vspace{1.5mm}
    \textbf{Definition}: \textit{A theorem is a proposition that is always true but not immediate to verify.}

\subsection*{Logical Operators}
    First, some notation:
    \begin{itemize}
        \item Propositions will generally be labeled as $p, q, r, \dots$.
        \item The negation of a proposition is indicated by the $\neg$ symbol. (ex. $\neg p$).
        \item The conjunction of two propositions is indicated by the symbol $\land$.
        \item The disjunction of two propositions is indicated by the symbol $\lor$.
        \item The statement ``If $p$ then $q$" can be represented by the notation $p \Rightarrow q$. This is also known as a conditional statement.
    \end{itemize}
    Using this newly defined notation, let's introduce a new proposition. $$ \text{For any natural number $n$, and for any set of integers $x$, $y$, and $z$, we have $x^n + y^n = z^n \Rightarrow xyz = 0$.}$$ The truth value of this proposition is clearly false (choose $n = 1$ and corresponding $x, y, z$). However, restricting the domain of $n$ to $n > 2$, this proposition becomes true (see Fermat's Last Theorem).
    
    \vspace{1.5mm}
    Truth tables can also be used to more easily determine the validity of a propositional statement, allowing us to brute-force the result by considering every possible combination of $p$, $q$, \dots, etc.
    $$\begin{array}{|c|c|c|c|c|c|}
        \hline
        p & q & \neg p & p \land q & p \lor q & p \Rightarrow q\\
        \hline
        T & T & F & T & T & T \\
        T & F & F & F & T & F \\
        F & T & T & F & T & T \\
        F & F & T & F & F & T \\
        \hline
    \end{array}$$

\subsection*{Direct Proofs}
    This is the first tool (of many) that we will use to prove propositional statements. The goal of a direct proof is to determine the truth value of a proposition with as few assumptions as possible. For example, when proving an implication of the form $p \Rightarrow q$, one would assume $p$ is true and attempt to prove $q$ is true, as shown below.

    \vspace{1.5mm}
    \textbf{Example}
    
    A natural number $n$ is said to be \textit{even} if there is another natural number $k$ such that $n = 2k$. Show that if $n$ and $m$ are even numbers, then $n + m$ is even.
    
    \vspace{1.5mm}
    \textbf{Solution}
    
    Suppose $n$ and $m$ are even numbers. Then there exists natural numbers $k$ and $l$ such that $n = 2k$ and $m = 2l$. Now $n + m = 2(k + l)$, so by definition $n + m$ is even.
    
\section*{Quantifiers and Predicates}
    Consider the following statements. Which ones are propositions? Why?
    \begin{enumerate}
        \item You will get an A in CS 2051 only if Pablo Picasso was a painter.
        \item $x > 3$.
        \item For all real numbers $x$, $x + 1 > x$.
        \item There exists a pair of two irrational numbers, $x, s$ such that $x^s$ is rational.
    \end{enumerate}

\end{document}