\documentclass{article}
\usepackage[margin=1in]{geometry}
\usepackage{amsmath, amssymb, amsthm}
\usepackage{enumitem}

% colored links
\usepackage{hyperref}
\hypersetup{
    colorlinks=true,
    linkcolor=blue,
    filecolor=magenta,      
    urlcolor=blue,
    }
% Inputting Python code
\usepackage[dvipsnames]{xcolor}
\definecolor{textblue}{rgb}{.2,.2,.7}
\definecolor{textred}{rgb}{0.54,0,0}
\definecolor{textgreen}{rgb}{0,0.43,0}
\usepackage{upquote}
\usepackage{listings}
\lstset{
    language=Python, 
    tabsize=4,
    basicstyle={\small\ttfamily},
    keywordstyle=\color{textblue},
    commentstyle=\color{textgreen},
    stringstyle=\color{textred},
    frame=none,
    columns=fullflexible,
    keepspaces=true,
    showstringspaces=false,
    xleftmargin=-1cm, % manual adjustment, figure out permanent solution
}

%Creating algorithms
\usepackage{algorithm}
\usepackage{algpseudocode}

\usepackage{tcolorbox}

%Images
\usepackage{graphicx}
    \usepackage{subcaption}
    \usepackage{float}
    % \usepackage[labelsep=period]{caption}

%Formatting and Spacing
\setitemize[1]{noitemsep, parsep = 5pt, topsep = 5pt}
\setenumerate[1]{label = (\alph*), parsep = 1pt, topsep = 5pt}
\setlength\parindent{0pt}
\linespread{1.1}

% title
\title{\vspace{-1cm}CS 2051: Honors Discrete Mathematics \\Spring 2023 Homework 9 Supplement}
\author{Gerandy Brito, Nithya Jayakumar, Sarthak Mohanty}
\date{}

\begin{document}

\maketitle

\section*{Question 1: Backward Dominoes}
    \textit{(Sarthak)} You may have heard the domino analogy used to describe induction (reword) where we assume the first domino falls, and the fall of the $n$th domino implies the fall of the $n + 1$ domino, then all the dominoes eventually fall. However, what if the dominoes fell \textit{backwards}? Suppose we have proven the following facts with respect to some predicate $P(n)$:
    \begin{gather}
        P(1) \\
        \forall n \in \mathbb{N}^{+}, P(n) \implies P(n - 1) \\
        \forall n \in \mathbb{N}, P(n) \implies P(kn)
    \end{gather}
    Now use induction to prove that $\forall k, n \in \mathbb{N}$, $P(n)$.

\section*{Question 2: }
    \textit{(Nithya)}

\section*{Question 3: }
    \textit{(Brito)} Let $S=\{s_1,s_2,\ldots, s_{2n-1}\}$ natural numbers, where $n=2^k>1$. Show that one can choose a subset $S'\subset S$ of cardinality $|S'|=n$ such that the sum of the elements in $S'$ is a multiple of $n$.\\

   (easier?) Let $a\in\mathbb{R}$ such that $a+1/a$ is an integer. Show that $a^n+1/a^n$ is an integer for all $n\in\mathbb{N}$.\\




\end{document}