\documentclass{article}
\usepackage[margin=1in]{geometry}
\usepackage{amsmath, amssymb, amsthm}
\usepackage{enumitem}

%Formatting and Spacing
\setitemize[1]{noitemsep, parsep = 5pt, topsep = 5pt}
\setenumerate[1]{label = (\alph*), parsep = 1pt, topsep = 5pt}
\setlength\parindent{0pt}
\linespread{1.1}

% title
\title{\vspace{-1cm}CS 2051: Honors Discrete Mathematics \\Spring 2023 Homework 8 Supplement}
\author{Gerandy Brito, Nithya Jayakumar, Sarthak Mohanty}
\date{}

\usepackage{tcolorbox}

\begin{document}

\maketitle

\section*{Question 1}
    \textit{(Warm-up)} The concept of induction is often compared to the domino analogy. It starts with the assumption that the first domino falls, and from there, we can infer that if the $n$th domino falls, then the $(n + 1)$th domino will also fall. This leads to the eventual falling of all the dominoes. But what if the dominoes fell in the opposite direction? 
    
    \vspace{2mm}
    Suppose we have proven the following facts with respect to some predicate $P(n)$:
    \begin{gather}
        P(1) \\
        \forall n \in \mathbb{N}^{+}, P(n) \implies P(n - 1) \\
        \forall n \in \mathbb{N}, P(n) \implies P(kn)
    \end{gather}
    Use induction to prove that $(\forall n,k \in \mathbb{N})(P(n))$.

\section*{Question 2 }
    \textit{(Brito)} Let $S=\{s_1,s_2,\ldots, s_{2n-1}\}$ natural numbers, where $n=2^k>1$. Show that one can choose a subset $S'\subset S$ of cardinality $|S'|=n$ such that the sum of the elements in $S'$ is a multiple of $n$.

% \section*{Sample Answer Format}

% \begin{tcolorbox}
%         Let $P(n)$ be the statement $$\text{insert statement here}.$$ \\
%         \textsc{Base Case:} We first prove $P(\cdot)$ is true.

%         \vspace{2mm}
%         \textsc{Inductive Hypothesis:} Now let $n \in \mathbb{N}$ such that $P(n)$\footnote{For strong induction, this would be $P(\cdot), P(\cdot + 1), \dots, P(n)$} is true. \dots. Therefore, $P(n + 1)$ is true as well.

%         \vspace{2mm}
%         \textsc{Conclusion:} Therefore by induction, for all $n \in \{\mathbb{N}, \mathbb{Z}, \text{etc}\}$, $P(n)$ is true.
% \end{tcolorbox}

\end{document}