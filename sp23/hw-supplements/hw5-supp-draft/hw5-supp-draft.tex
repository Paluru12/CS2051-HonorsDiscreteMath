\documentclass{article}
\usepackage[margin=1in]{geometry}
\usepackage{amsmath, amssymb, amsthm}
\usepackage{enumitem}

%Highlighting
\usepackage{xcolor, soul}
\sethlcolor{lightgray}

%Cases Environment
\newlist{Cases}{enumerate}{3}
\setlist[Cases]{leftmargin = .25in, label = {Case \arabic*.}, topsep = 0.01in, itemsep = 0.04in, itemindent = .5in, parsep = 0in}

%Formatting and Spacing
\setitemize[1]{noitemsep, parsep = 5pt, topsep = 5pt}
\setenumerate[1]{label = (\alph*), parsep = 1pt, topsep = 5pt}
\setlength\parindent{0pt}
\linespread{1.15}

%Formatting and Spacing
\usepackage{enumitem}
\setitemize[1]{noitemsep, parsep = 5pt, topsep = 5pt}
\setenumerate[1]{label = (\alph*), parsep = 1pt, topsep = 5pt}
\setlength\parindent{0pt}
\linespread{1.1}

% title
\title{\vspace{-1cm}CS 2051: Honors Discrete Mathematics \\Spring 2023 Homework 5 Supplement}
\author{Sarthak Mohanty }
\date{}

\begin{document}

\maketitle


\textbf{Overview}
    Traditionally, computer software has been written for serial computation: instructions corrresponding to a tasks are executed on one processing unit at a time.



  computational task consisted of many elementary operations, some of which had to be computed sequentially, while others could be computed in parallel. 

  In a distributed computing class, one of the first things you learn is \textbf{Amdahl’s law}, which gives a quantitative measure of the speedup $S$ -i.e., how much faster the task can be performed by using $n$ parallel processors:
  $$S = \frac{1}{1 - p + \frac{p}{n}}.$$ Here, $p$ is the proportion of operations that can be performed in parallel.

    \vspace{2mm}
    However, it is not typically the case that we can classify operations simply as “parallel” or “sequential.” Instead, a task might consist of several sub-tasks, some of which need to be completed before others are started. The ordering of these subtasks, as well as computing the speedup, is a much harder problem (comparable in difficulty to the \verb+SAT+ problem encountered previously).

    \vspace{2mm}
    In this supplement, you'll develop the mathematical intuition to formally represent this problem. You've learned about functions, now you'll learn about a specialized version of them known as \textbf{relations}.
    
    
    You'll then apply different techniques to solve and analyze different applications of this problem.





\section*{Relations}

    Recall the definition of a relation:
    
    \vspace{1.5mm}
    \textbf{Definition} a \textit{relation} is a subset of the Cartesian product $A\times B$.
    
    \vspace{1.5mm}
    We denote relations by $\mathcal{R}$. We write $a\mathcal{R} b$ to indicate that $(a,b)\in \mathcal{R}$ (i.e.: in the subset denoted by $\mathcal{R}$). When $A=B$ we said that $\mathcal{R}$ is a relation on $A$.\\
    
    Let $\mathcal{R}_1, \dots, \mathcal{R}_4$ be a relation on $A = \{1, 2, 3, 4\}$.
    \begin{itemize}
        \item $\mathcal{R}_1 = \{(a, b) \mid a \le b)\}$
        \item $\mathcal{R}_2 = \{(a, b) \mid a = b)\}$
        \item $\mathcal{R}_3 = \{(a, b) \mid a+b \le 2022)\}$
        \item $\mathcal{R}_4 = \{(a, b) \mid a \text{ divides } b\}$
    \end{itemize}
    So why relations? They are more general and allow us to study more complex sets. Elaborate. \\
    
\subsection*{Properties}
    
    \begin{itemize}
        \item \underline{Reflexive:} $(\forall a \in A)(a\mathcal{R}a)$
        \item \underline{Symmetric:} $(\forall a, b \in A)(a\mathcal{R}b \iff b\mathcal{R}a)$
        \item \underline{Antisymmetric:} $(\forall a, b\in A)(a\mathcal{R}b \wedge b\mathcal{R}a \rightarrow a=b)$
        \item \underline{Transitive:} $(\forall a, b, c\in A)(a\mathcal{R}b \wedge b\mathcal{R}c \rightarrow a\mathcal{R}c)$
    \end{itemize}

    Some examples of relations:


\section*{Partially Ordered Sets}

\end{document}