\documentclass{article}
\usepackage[margin=1in]{geometry}
\usepackage{amsmath, amssymb, amsthm}
\usepackage{enumitem}

% colored links
\usepackage{hyperref}
\hypersetup{
    colorlinks=true,
    linkcolor=blue,
    filecolor=magenta,      
    urlcolor=blue,
    }



% Inputting Python code
\usepackage[dvipsnames]{xcolor}
\definecolor{textblue}{rgb}{.2,.2,.7}
\definecolor{textred}{rgb}{0.54,0,0}
\definecolor{textgreen}{rgb}{0,0.43,0}
\usepackage{upquote}
\usepackage{listings}
\lstset{
    language=Python, 
    tabsize=4,
    basicstyle={\ttfamily},
    keywordstyle=\color{textblue},
    commentstyle=\color{textgreen},
    stringstyle=\color{textred},
    frame=none,
    columns=fullflexible,
    keepspaces=true,
    showstringspaces=false,
    xleftmargin=-15mm, % manual adjustment, figure out permanent solution
}
\usepackage{tcolorbox}
\tcbuselibrary{skins,hooks}
\usetikzlibrary{shadows}
\usepackage{lipsum}

%Images
\usepackage{graphicx}
    \usepackage{subcaption}
    \usepackage{float}

%Formatting and Spacing
\setitemize[1]{noitemsep, parsep = 5pt, topsep = 5pt}
\setenumerate[1]{label = (\alph*), parsep = 1pt, topsep = 5pt}
\setlength\parindent{0pt}
\linespread{1.1}

% title
\title{\vspace{-1cm}CS 2051: Honors Discrete Mathematics \\Spring 2023 Homework 7 Supplement}
\author{Sarthak Mohanty }
\date{}

\begin{document}

\maketitle


Title: Language Building with Recursion

\section*{Overview}

% https://web.stanford.edu/class/archive/cs/cs103/cs103.1234/timeline_of_results
If you ever peruse an article about important results in discrete math, you'll often see a statement similar to ``Noam Chomsky publishes the book “Syntactic Structures."  Noam Chomsky may be many things, including a world-renowned linguist and psychologist, but it seems difficult to see his correlation to discree mthat.

In 2050, you covered induction, a way to ... . In this supplement, we'll cover recursion, and delve into the intimate counterplay between induction and recursion. we'll introduce computational models, including an important one called a context-free grammar. You'll make your own called a parser,.

\section*{What is Recursion}

If you already have some knowledge of recursion, you can skip this exposition.
\subsection*{Induction vs Recursion}


\section*{Modeling Computation}

% https://courses.engr.illinois.edu/cs173/fa2016/B-lecture/Lectures/Induction%20proof%20on%20Context%20Free%20Grammars.pdf

\section*{Syntactical Structures}

\section*{Designing English}



% \section*{(Optional) Context-Sensitive Grammars}

% textbook on engilsh as cfg: 

% https://www.google.com/search?q=generalized+phrase+structure+grammar&oq=Generalized+Phrase+Structure+Grammar&aqs=chrome.0.0i512l2j0i15i22i30j0i390l3.452j0j7&sourceid=chrome&ie=UTF-8


% https://www.jstor.org/stable/4178381?seq=1

example parser:

% https://i.stack.imgur.com/2gOx4.jpg
\end{document}