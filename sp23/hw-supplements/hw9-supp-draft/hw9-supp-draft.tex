\documentclass{article}
\usepackage[margin=1in]{geometry}
\usepackage{amsmath, amssymb, amsthm}
\usepackage{enumitem}

% colored links
\usepackage{hyperref}
\hypersetup{
    colorlinks=true,
    linkcolor=blue,
    filecolor=magenta,      
    urlcolor=blue,
    }
% Inputting Python code
\usepackage[dvipsnames]{xcolor}
\definecolor{textblue}{rgb}{.2,.2,.7}
\definecolor{textred}{rgb}{0.54,0,0}
\definecolor{textgreen}{rgb}{0,0.43,0}
\usepackage{upquote}
\usepackage{listings}
\lstset{
    language=Python, 
    tabsize=4,
    basicstyle={\small\ttfamily},
    keywordstyle=\color{textblue},
    commentstyle=\color{textgreen},
    stringstyle=\color{textred},
    frame=none,
    columns=fullflexible,
    keepspaces=true,
    showstringspaces=false,
    xleftmargin=-1cm, % manual adjustment, figure out permanent solution
}

%Creating algorithms
\usepackage{algorithm}
\usepackage{algpseudocode}

\usepackage{tcolorbox}

%Images
\usepackage{graphicx}
    \usepackage{subcaption}
    \usepackage{float}
    % \usepackage[labelsep=period]{caption}

%Formatting and Spacing
\setitemize[1]{noitemsep, parsep = 5pt, topsep = 5pt}
\setenumerate[1]{label = (\alph*), parsep = 1pt, topsep = 5pt}
\setlength\parindent{0pt}
\linespread{1.1}

% title
\title{\vspace{-1cm}CS 2051: Honors Discrete Mathematics \\Spring 2023 Homework 9 Supplement}
\author{Sarthak Mohanty}
\date{}

\begin{document}

\maketitle

\section*{Backward Dominoes}
    You may have heard the domino analogy used to describe induction (insert description here). However, what if the dominoes fell \textit{backwards}? Suppose we have proven the following facts with respect to some predicate $P(n)$:
    \begin{gather}
        P(1) \\
        \forall n \in \mathbb{N}^{+}, P(n) \implies P(n - 1) \\
        \forall n \in \mathbb{N}, P(n) \implies P(2n)
    \end{gather}
    In this question, you will show that, taken together, these three statements comprise a valid proof that $P$ holds for all naturals
    \begin{enumerate}
        \item Use complete induction to prove that $\forall n \in \mathbb{N}$, $P(n)$.
        \item If we failed to prove (3), but kept the other two statements, what values would we be able to conclude that $P$ holds for? Repeat for (2) and (1).
    \end{enumerate}

\section*{Structural Induction}




\end{document}