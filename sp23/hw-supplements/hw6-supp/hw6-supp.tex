\documentclass{article}
\usepackage[margin=1in]{geometry}
\usepackage{amsmath, amssymb, amsthm}
\usepackage{enumitem}

%Highlighting
\usepackage{xcolor, soul}
\sethlcolor{lightgray}

%Cases Environment
\newlist{Cases}{enumerate}{3}
\setlist[Cases]{leftmargin = .25in, label = {Case \arabic*.}, topsep = 0.01in, itemsep = 0.04in, itemindent = .5in, parsep = 0in}

%Formatting and Spacing
\setitemize[1]{noitemsep, parsep = 5pt, topsep = 5pt}
%\setenumerate[1]{label = (\alph*), parsep = 1pt, topsep = 5pt}
\setlength\parindent{0pt}
\linespread{1.15}

% title
\title{\vspace{-1cm}CS 2051: Honors Discrete Mathematics \\Spring 2023 Homework 6 Supplement}
\author{Nithya Jayakumar }
\date{}

\begin{document}

\maketitle
\begin{enumerate}
    \item A countable set is a set that has a one-to-one mapping with the set of natural numbers. Prove that the set of positive rational numbers is countable by setting up a function that assigns to a rational number $p/q$ with $\gcd (p, q) = 1$ the base 11 number formed by the decimal representation of p followed by the base 11 digit A, which corresponds to the decimal number 10, followed by the decimal representation of q. 

    \item Define a Carmichael number as a composite number n which satisfies the following relation: $b^n \equiv b \pmod{n},$ for all integers b.
    Show that if $n = p_1p_2 \cdots p_k$, where $p_1, p_2, \dots, p_k$ are distinct primes that satisfy $p_j - 1 \vert n - 1$ for $j = 1, 2, \dots, k,$ then $n$ is a Carmichael number.
\end{enumerate}

\end{document}