\documentclass{article}

%Basics
\usepackage[margin=1in]{geometry}
\usepackage{amsmath, amssymb, amsthm}
\usepackage{enumitem}

%Formatting and Spacing
\setitemize[1]{noitemsep, parsep = 5pt, topsep = 5pt}
\setitemize[2]{noitemsep, parsep = 5pt, topsep = 5pt}
\setenumerate[1]{label = (\alph*), parsep = 1pt, topsep = 5pt}
\setenumerate[2]{label = \roman*., parsep = 1pt, topsep = 5pt}
\setlength\parindent{0pt}
\linespread{1.1}

\begin{document}

\section*{Outline}

In this project, you will have the opportunity to delve into a topic of your choice related to discrete mathematics and demonstrate your understanding and mastery of the material covered in the course.

\vspace{3mm}
You will be using \LaTeX\ to typeset a mathematical document that explores a topic, develops all the necessary definitions and preliminary results, and applies the main result to an interesting or important problem or generalizes or extends the main result in a new direction.
    
\section*{Rubric}
Your project will be graded out of a total of 100 points, according to the following criteria, whose weightings are indicated:

\begin{itemize}
    \item \textbf{Project Outline} (15 points): To receive full credit, your project outline should include your main idea, how you will communicate your main idea, and how you will take this idea further with generalizations. You should use the project outline template.
    \item \textbf{Project Draft} (15 points): At this stage, you should expand on the results developed and move your results from an outline format to a formalized document. By this point, you should have most of your results proven and illustrated. Mentors will review your draft and note chnages that should be made and possible improvements.
    \item \textbf{Final Paper} (30 points):
    \begin{itemize}
        \item \textbf{Comprehensibility and examples.}
        Your document should be self-contained, but may assume knowledge of the topics we have covered in this course. Any other definitions and preliminary results needed to understand the statement and proof of your main result, and material in subsequent sections, should be developed in your project.
        
        Your document should be comprehensible to another student in the course, and should contain examples illustrating the concepts and results covered.
        \item \textbf{Mathematical correctness.} The definitions, result statements and proofs in your document should all be mathematically correct, and should be written in enough detail that another student in the course can understand them.
        \item \textbf{Accuracy of mathematical writing.} Your use of mathematical notation and terminology should be accurate, and your variables should be correctly quantified.
    
        \textbf{Formatting.} Your document should be laid out neatly and should look professional. It should be typeset in \LaTeX, and should use the templates we have provided. A more specific breakdown of this criterion is as follows.
        \begin{itemize}
            \item \textbf{Accuracy and appearance}: the extent to which the typeset document is neat, professional and readable.
            \item \textbf{Document structure}: use of sections and subsections, paragraphs, and both bulleted and enumerated lists.
            \item \textbf{Text formatting}: use of emphasis (e.g. bold face, italic or underlined text), text alignment and different fonts.
            \item \textbf{Mathematical notation}: use of math mode to typeset mathematical notation, appropriate use of variables and symbols, in-line and displayed equations, and aligned equations.
            \item \textbf{Results, definitions and references}: appropriate use of definition, theorem and proof environments, and use of labels and references.
        \end{itemize}
    \end{itemize}
    \textbf{Final Presentation} (30 points): The final presentation will be a science fair-style affair, with groups setting up posters or presentations in a room with the intention of presenting their findings and ideas to their peers and instructors. Nithya will be in charge of this.
\end{itemize}


\end{document}