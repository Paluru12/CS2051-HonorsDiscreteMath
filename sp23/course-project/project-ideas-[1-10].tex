\documentclass{article}
\usepackage[margin=1in]{geometry}
\usepackage{amsmath, amssymb, amsthm}
\usepackage{enumitem}

% colored links
\usepackage{hyperref} 
\hypersetup{
    colorlinks=true,
    linkcolor=blue,
    filecolor=magenta,      
    urlcolor=blue,
    }

%Formatting and Spacing
\setitemize[1]{noitemsep, parsep = 5pt, topsep = 5pt}
\setenumerate[1]{label = (\alph*), parsep = 1pt, topsep = 5pt}
\setlength\parindent{0pt}
\linespread{1.1}

% title
\title{\vspace{-1cm}CS 2051: Honors Discrete Mathematics \\Spring 2023}
\author{}
\date{}

\begin{document}

\maketitle

Requirements for the project:


\bigskip
Nithya's:
%Some projects are dealing with advanced ideas. Understanding, even partially, one such proofs is considered satisfactory work. Students should aim to offer classmates an overview of the problem, some ideas of the proof and anything else they are capable of achieving.
\begin{enumerate}[label = \arabic*.]
    \item The real numbers as the (topological) closure of the rationals. Explore the notion of {\it field of numbers} and their applications. 

        The rationals are defined as $\bb Q = \{x/y : x,y\in\bb Z\},$ or the set of all fractions of integers.
        This contains numbers like $1/2$, $-52/3,$ and $100,$ but not numbers like $\sqrt 2$ or $\pi$ or $0.110100100010000\ldots$
        In order to reach those numbers, we can take one main approach to closure: a sequence of rational numbers $(a_i)$ corresponds to a unique real number if it has a limit.

        How do we define a sequence having a limit?
        You have probably heard of epsilon-delta proofs for the existence of limits.
        In our case, we will use an epsilon-N proof to define the sequence of rational numbers $(a_i)$ approaching a limit of a real number $r.$
        The following statement is the statement we need to use to prove that such a sequence approaches a limit:
        $$\forall\epsilon\exists\N\forall n\in\bb Z ((\epsilon>0\land n>N) \to |a_n-r|<\epsilon).$$

        We assert that a real number exists if there is some sequence approaching it.
        For most real numbers that we're interested in, this is pretty easy because we have a decimal expansion.
        For example, we associated $\sqrt 2$ with $\{1,1.4,1.41,1.414,1.4142,1.41421,1.414213,1.4142135,\ldots\},$ so this number exists!

        This is called a topological closure because topology is the tool that we use to handle points being arbitrarily close to each other without actually touching.
        
        You will investigate how we can define the real numbers in terms of topology during your project.

        The real numbers are in a category of objects known as fields: objects with a notion of commutative addition, commutative multiplication (related by the distributive property $a\cdot (b+c)=a\cdot b + a\cdot c$), a $0$ element, a $1$ element, and division and subtraction.
        The complex numbers and the rationals are also fields, but there are also some more exotic objects: the finite fields $\bb F_p$ only have a finite number of elements, but we can use all of the field theorems we've learned on them!

        You will investigate these.

        Ideas for directions to go in:
        \begin{itemize}
            \item Abel-Ruffini Theorem: there is no solution to radicals to general polynomial equations of degree five or higher with arbitrary coefficients. 
            \item Cayley's Theorem: every group is isomorphic to the group of permutations of n objects for some number.
            \item Gauss's Theorem (primitivity): If P(x) and Q(x) are primitive polynomials over the integers, their product P(x)Q(x) is also primitive.
            \item In the same way that vector spaces are defined over fields, modules are defined over rings. Investigate the ways that the properties of vector spaces like the Basis Theorem might extend to modules.
        \end{itemize}
    \item What it means to be irrational? Proofs and extensions. %transcendental numbers!
        %https://mathworld.wolfram.com/LagrangeNumber.html
        %https://mathworld.wolfram.com/HurwitzsIrrationalNumberTheorem.html
        

        \begin{itemize}
            \item Is $\pi + e$ irrational?
            \item Gelfond-Schneider Theorem: If a and b are algebraice numbers with $a \neq 0$ and $1$ and $b$ are irrational, then any value of $a^b$ is a transcendental number.
        \end{itemize}
        
    \item Probability on discrete settings is like counting: Catalan numbers, Dyck's paths and The Semicircle Law (*)
        
    \item Ordered Sets
        
    \item Prime numbers and big theorems: Tao and Green's, Zhang's, Helfgott's. (**)
    \item Group of permutations and its combinatorics.
        
    \item Sylow Theorems and its applications (*)
        
    \item Integer Partitions
        
    \item Carmichael Numbers (https://www.quantamagazine.org/teenager-solves-stubborn-riddle-about-prime-number-look-alikes-20221013/)
        
    \item Four Color Theorem (**)
        
    \item https://projecteuler.net/problem=670
    \item https://projecteuler.net/problem=676
\end{enumerate}

(*) Hard. Will require more research to understand the theorem and its implications, but we don't require as much demo of the concepts.
(**) Extremely hard. If you just understand how to prove the theorem and why we care about it, we will be happy with your project!

\end{document}