\documentclass{article}
\usepackage[margin=1in]{geometry}
\usepackage{amsmath, amssymb, amsthm}
\usepackage{enumitem}

% colored links
\usepackage{hyperref} 
\hypersetup{
    colorlinks=true,
    linkcolor=blue,
    filecolor=magenta,      
    urlcolor=blue,
    }

%Formatting and Spacing
\setitemize[1]{noitemsep, parsep = 5pt, topsep = 5pt}
\setenumerate[1]{label = (\alph*), parsep = 1pt, topsep = 5pt}
\setlength\parindent{0pt}
\linespread{1.1}

% title
\title{\vspace{-1cm}CS 2051: Honors Discrete Mathematics \\Spring 2023}
\author{}
\date{}

\begin{document}

\maketitle

Each TA is in charge of 10 ideas, for a total of 20 project ideas

Nithyas:
%Some projects are dealing with advanced ideas. Understanding, even partially, one such proofs is considered satisfactory work. Students should aim to offer classmates an overview of the problem, some ideas of the proof and anything else they are capable of achieving.
\begin{enumerate}[label = \arabic*.]
    \item The real numbers as the (topological) closure of the rationals. Explore the notion of {\it field of numbers} and their applications. 
    \item What it means to be irrational? Proofs and extensions. %trascendental numbers!
    \item Probability on discrete settings is like counting: Catalan numbers, Dyck's paths and The Semicircle Law(*)  
    \item Ordered Sets (?)
    \item Prime numbers and big theorems: Tao and Green's, Zhang's, Helfgott's.(**) 
    \item Group of permutations and its combinatorics.(?)
    \item Sylow Theorems and its applications (*)
    \item Integer Partitions
    \item Carmichael Numbers (https://www.quantamagazine.org/teenager-solves-stubborn-riddle-about-prime-number-look-alikes-20221013/)
    \item Four Color Theorem (**)
    \item https://projecteuler.net/problem=670
    \item https://projecteuler.net/problem=676
\end{enumerate}

\pagebreak



Sarthaks:
\begin{enumerate}[label = \arabic*.]
    \item Cryptography in the Modern Age: As we have seen (or will see), security standards are being changed for various reasons. Explore some of the 
    \item Recursive Functions: A standard tool in any programmer's toolbox. Main ideas could be master theorem and fractals.
    \item Category Theory / Godel's Incompleteness Theorem. \textcolor{cyan}{a student told me they will do this project! GB}
    \item Lambda Calculus
    \item Game Theory: Examples of mathematical games to analyze include Conway's Game of Life, Sprouts, or Dots and Boxes (Sprague–Grundy theorem)
    \item Make your own Game!
    
    You can also make your own game, have students play the game, and try to find a solution.
    \item Monte Carlo method
\end{enumerate}

\pagebreak

\section*{Example}

\subsection*{Linear Diophantine Equations}
    A \textit{linear Diophantine equation} is an equation of the form $$a_{1}x_{1} + a_{2}x_{2} + \dots + a_{n}x_{n} = c$$ where $a_{1}, a_{2}, \dots, a_{n}, c \in \mathbb{Z}$.
    
    \vspace{3mm}
    In the course, we considered the case when $n = 2$ and asked when the equation $ax + by = c$ has an integral solution. We proved that a solution exists if and only if $\gcd(a, b) \mid c$, or else no solution exists -- this was B\'ezout's Lemma. However, we didn't develop a way of \textit{finding} the solution.
    
    \vspace{3mm}
    One method to find a solution is to reverse the Euclidean algorithm (see page 131 in the text). The aim of this project is to classify \textit{all} solutions to such an equation. This leads us to the following main result:
    
    \vspace{3mm}
    \textbf{Theorem.} Let $a, b, c \in \mathbb{Z}$, let $d = \gcd(a, b)$ and suppose that $d \mid c$. Let $(x_{0}, y_{0})$ be a fixed integral solution to the equation $ax + by = c$. A pair of integers $(x, y)$ is a solution to the equation if and only if $$x = x_{0} + k \cdot \frac{b}{d} \text{\quad and\quad} y = y_{0} - k \cdot \frac{a}{d}$$ for some $k \in \mathbb{Z}$.
    
    \vspace{3mm}
    A couple of possible areas for extension or generalizations are as follows:
    \begin{itemize}
        \item Consider the case where the coefficients and solutions to the equation $ax + by = c$ are required to be \textit{natural numbers} -- that is, we fix $a, b, c \in \mathbb{N}$ and ask when natural number solutions $x, y$ to the equation $ax + by = c$ exist. When $a$ and $b$ are coprime, it is know that the least value of $c$ for which no natural number solution exists is $ab - a - b$. Prove this, and consider what happens when $a$ and $b$ are \textit{not} coprime.
        \item Consider the case when there are $n \ge 1$ variables, not necessarily exactly two variables - that is, consider an equation of the form $\sum_{k = 1}^{n} a_{k}x_{k} = c$ for some $n \ge 1$ and some $a_{1}, a_{2}, \dots, a_{n}, c \in \mathbb{Z}$. Generalize B\'ezout's lemma to this case; this will require that you define a notion of greatest common divisor for collections of $n$ integers, and then find a prove a necessary and sufficient condition for the equation $\sum_{k = 1}^{n} a_{k}x_{k} = x$ to have a solution.
    \end{itemize}
    


\end{document}