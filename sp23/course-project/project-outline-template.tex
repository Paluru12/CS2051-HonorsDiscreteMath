\documentclass{article}
\usepackage[margin=1in]{geometry}
\usepackage{amsmath, amssymb, amsthm}
\usepackage{enumitem}

\title{CS 2051: Project Title}
\author{Author 1 \\ Georgia Institute of Technology
\and Author 2 \\ Georgia Institute of Technology
\and Author 3 \\ Georgia Institute of Technology}
\date{}

%%%%%%%%%%%%%%%%%%%%%%%%%%%%%%%%%%%%%%%%%%%%%%%%%%%%%%%%%%%%%%%%

\begin{document}

\maketitle


\section{Background}
% This section should develop the basic definitions and
% preliminary results required for the statement and proof of
% your main result, including examples that help the reader to
% develop intuition for the concepts presented.



\section{Main result}
% In this section, you should state and prove your main result,
% and provide some basic consequences and examples that help the
% reader to understand it. You may want to change this section's
% name to something more informative.



\section{Extension/application/generalisation}
% You should change this section's name to something relevant to
% your project (e.g. "Applications of the RSA encryption
% algorithm" or "Linear Diophantine equations in $n$ unknowns"



\section{Reflection/conclusion/future directions}
% You should change this section's name to something relevant to
% your project. This section allows you to reflect or conclude
% the work you have done in your project and discuss future work
% that could be done on the project (e.g. applications you tried
% to make but couldn't find the time).



\section{Acknowledgements and references}
% Here you should acknowledge people whose help you are thankful
% for (and why), and any sources such as books and websites that
% you used when studying for the project.


\pagebreak


\section*{(Informal) Example Outline}

\subsection*{Linear Diophantine Equations}
    A \textit{linear Diophantine equation} is an equation of the form $$a_{1}x_{1} + a_{2}x_{2} + \dots + a_{n}x_{n} = c$$ where $a_{1}, a_{2}, \dots, a_{n}, c \in \mathbb{Z}$.
    
    \vspace{3mm}
    In the course, we considered the case when $n = 2$ and asked when the equation $ax + by = c$ has an integral solution. We proved that a solution exists if and only if $\gcd(a, b) \mid c$, or else no solution exists -- this was B\'ezout's Lemma. However, we didn't develop a way of \textit{finding} the solution.
    
    \vspace{3mm}
    One method to find a solution is to reverse the Euclidean algorithm (see page 131 in the text). The aim of this project is to classify \textit{all} solutions to such an equation. This leads us to the following main result:
    
    \vspace{3mm}
    \textbf{Theorem.} Let $a, b, c \in \mathbb{Z}$, let $d = \gcd(a, b)$ and suppose that $d \mid c$. Let $(x_{0}, y_{0})$ be a fixed integral solution to the equation $ax + by = c$. A pair of integers $(x, y)$ is a solution to the equation if and only if $$x = x_{0} + k \cdot \frac{b}{d} \text{\quad and\quad} y = y_{0} - k \cdot \frac{a}{d}$$ for some $k \in \mathbb{Z}$.
    
    \vspace{3mm}
    A couple of possible areas for extension or generalizations are as follows:
    \begin{itemize}
        \item Consider the case where the coefficients and solutions to the equation $ax + by = c$ are required to be \textit{natural numbers} -- that is, we fix $a, b, c \in \mathbb{N}$ and ask when natural number solutions $x, y$ to the equation $ax + by = c$ exist. When $a$ and $b$ are coprime, it is know that the least value of $c$ for which no natural number solution exists is $ab - a - b$. Prove this, and consider what happens when $a$ and $b$ are \textit{not} coprime.
        \item Consider the case when there are $n \ge 1$ variables, not necessarily exactly two variables - that is, consider an equation of the form $\sum_{k = 1}^{n} a_{k}x_{k} = c$ for some $n \ge 1$ and some $a_{1}, a_{2}, \dots, a_{n}, c \in \mathbb{Z}$. Generalize B\'ezout's lemma to this case; this will require that you define a notion of greatest common divisor for collections of $n$ integers, and then find a prove a necessary and sufficient condition for the equation $\sum_{k = 1}^{n} a_{k}x_{k} = x$ to have a solution.
    \end{itemize}



\end{document}