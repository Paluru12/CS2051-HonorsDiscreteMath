\documentclass{article}
\usepackage[margin=1in]{geometry}
\usepackage{amsmath, amssymb, amsthm}
\usepackage{enumitem}

\title{CS 2051: Project Title}
\author{Author 1 \\ Georgia Institute of Technology
\and Author 2 \\ Georgia Institute of Technology
\and Author 3 \\ Georgia Institute of Technology}
\date{}

%%%%%%%%%%%%%%%%%%%%%%%%%%%%%%%%%%%%%%%%%%%%%%%%%%%%%%%%%%%%%%%%

\begin{document}

\maketitle

\section{Background}
% This section should develop the basic definitions and
% preliminary results required for the statement and proof of
% your main result, including examples that help the reader to
% develop intuition for the concepts presented. Start from scratch, remember your audience is other 2051 students.


\section{Main result}
% In this section, you should state your main result,
% and provide some basic consequences and examples that help the
% reader to understand it. You may want to change this section's
% name to something more informative.


\section{Extension/application/generalisation}
% You should change this section's name to something relevant to
% your project (e.g. "Applications of the RSA encryption
% algorithm" or "Linear Diophantine equations in $n$ unknowns"


\section{Preliminary Code and Illustrations}
% Here should be some illustration of the concepts described above. Also place any code you used to generate any illustrations (if this applies to you, you can use the lstlistings package).


\section{Reflection/Conclusion}
% You should change this section's name to something relevant to
% your project. This section allows you to reflect or conclude
% the work you have done in your project and discuss future work
% that could be done on the project (e.g. applications you tried
% to make but couldn't find the time).


\section{References}
% Here you should acknowledge people whose help you are thankful
% for (and why), and any sources such as books and websites that
% you used when studying for the project.
    

 
\end{document}