\documentclass{article}

%Basics
\usepackage{amsmath, amssymb, amsthm}
\usepackage{enumitem}
\usepackage[margin=1in]{geometry}


\usepackage{newpxtext} \usepackage{eulerpx}

\newcommand*{\myfont}{\fontfamily{qhv}\selectfont}
\DeclareTextFontCommand{\blackboardtext}{\myfont}
% \usepackage{eulerpx}

% \usepackage{fontspec}
% \newfontfamily{\differenttt}{Courier New}
% \DeclareTextFontCommand{\textdifferenttt}{\differenttt}

% colored links
\usepackage{hyperref}
\hypersetup{
    colorlinks=true,
    linkcolor=blue,
    filecolor=magenta,      
    urlcolor=magenta,
    }

%Creating algorithms
\usepackage{algorithm}
\usepackage[noend]{algpseudocode}

% Inputting Python code
\usepackage[dvipsnames]{xcolor}
\definecolor{textblue}{rgb}{.2,.2,.7}
\definecolor{textred}{rgb}{0.54,0,0}
\definecolor{textgreen}{rgb}{0,0.43,0}
\usepackage{upquote}
\usepackage{listings}
\lstset{
    language=Python, 
    tabsize=4,
    basicstyle={\ttfamily},
    keywordstyle=\color{textblue},
    commentstyle=\color{textgreen},
    stringstyle=\color{textred},
    frame=none,
    columns=fullflexible,
    keepspaces=true,
    showstringspaces=false,
    xleftmargin=-15mm, % manual adjustment, figure out permanent solution
}

% Tcolorbox
\usepackage[most]{tcolorbox}
\tcbuselibrary{skins,hooks}
\usetikzlibrary{shadows}
\tcbuselibrary{skins}
\usetikzlibrary{shadows.blur}

%Images
\usepackage{graphicx}
    \usepackage{subcaption}
    \usepackage{float}

%Creating Figures
\usepackage{tikz}
\usetikzlibrary{calc, math, matrix, graphs, positioning}
\usetikzlibrary{fit}
\usetikzlibrary{backgrounds}
\usetikzlibrary{hobby}

%Formatting and Spacing
\setitemize[1]{noitemsep, parsep = 5pt, topsep = 5pt}
\setenumerate[1]{label = (\alph*), parsep = 1pt, topsep = 5pt}
\setlength\parindent{0pt}
\linespread{1.1}

%Custom Title Fields
\newcommand{\hmwkTitle}{Extra Problems}
\newcommand{\hmwkDueDate}{Friday, April 28}
\newcommand{\hmwkClass}{Honors Discrete Mathematics}
\newcommand{\hmwkClassInstructor}{Professor Gerandy Brito}
\newcommand{\hmwkSection}{Spring 2023}
\newcommand{\hmwkAuthorName}{CS 2051 TAs}

%Headers and Footers
\usepackage{fancyhdr}
\usepackage{extramarks}
\pagestyle{fancy}
\lhead{CS 2051}
\chead{\hmwkClass \ (\hmwkClassInstructor)}
\rhead{\hmwkTitle}
\cfoot{\thepage}
\renewcommand\headrulewidth{0.4pt}
\renewcommand\footrulewidth{0.4pt}

\title{
    \vspace{2in}
    \textbf{\hmwkClass:\\ \hmwkTitle} \\
    \normalsize\vspace{0.1in}\small{Due\ on\ \hmwkDueDate\ at 11:59pm} \\
    \vspace{0.1in}\large{\textit{\hmwkClassInstructor\ \hmwkSection}} \\
    \vspace{1in}
    \begin{minipage}[t]{0.5\columnwidth}
    {\footnotesize \textbf{IMPORTANT:} For this assignment, you may NOT collaborate with other students.}
    \end{minipage}
    \vspace{1in}
    \author{\textbf{\hmwkAuthorName}}
    \date{}
}

\begin{document}

\maketitle
\pagebreak


\section*{Challenge Problem}
    \subsection*{Instructions}
    Each semester, we release a challenge problem. It's ungraded, but your submission/discussion of this problem will be considered if you ever apply to be a TA for this course. You can find the previous years' problems \href{https://github.com/sar-mo/CS2051-HonorsDiscreteMath/blob/main/sp22/extra-credit/Exam 2 Extra Credit.pdf}{here}. 

    \subsection*{The Problem}
    In the Homework 5 Supplement, we wished to find a solution to the \texttt{Task} \texttt{Scheduling} problem, which we restate here:

    \begin{tcolorbox}[
    enhanced, colback={green!40!black}, colupper=white, colframe=brown!70!black,
    sharp corners,drop fuzzy shadow,
    underlay={\tcbvignette{size=2mm, inside node=frame, raised color=brown!70!black}}]
    {
        \vspace{2mm} We are given a partially ordered set of tasks $T = {T_{1}, T_{2}, \dots, T_{n}}$, where each task $T_{i}$ takes 1 unit of time to complete. The tasks are partially ordered by a binary relation $\prec$, which specifies the dependencies between the tasks. Specifically, if $T_{i} \prec T_{j}$, then $T_{i}$ must be completed before $T_{j}$ can be started.
    
        \vspace{3mm}
        We also have $k$ processors $P_{1}, P_{2}, \dots, P_{k}$ available to perform the tasks. A \textbf{schedule} for $T$ assigns each task $T_{i}$ to a processor $P_{j}$ and a start time $t_{i}$ for $P_{j}$ to begin working on $T_{i}$. If a processor starts working on $T_{i}$ at time $t_{i}$, then $T_{i}$ will be completed at time $t_{i} + 1$.
    
        \vspace{3mm} A schedule is \textbf{feasible} if it satisfies the following conditions:
        \begin{enumerate}[label = \arabic*]
            \item No single processor performs multiple tasks at the same time. In other words, if a processor is assigned to both $T_{i}$ and $T_{j}$, where $i \ne j$, then the start times for the two tasks must be different.
            \item For every pair of tasks $T_{i}$ and $T_{j}$ such that $T_{i} \prec T_{j}$, the start time for $T_{j}$ is later than or equal to the completion time for $T_{i}$. In other words, if $T_{i}$ must be completed before $T_{j}$ can start, then the start time for $T_{j}$ must be after or at the same time as the completion time for $T_{i}$.
        \end{enumerate}
        The \textbf{latency} of a schedule is defined as the time between the earliest start time for any task and the latest completion time for any task. Our goal is to find a feasible schedule with the smallest possible latency.
        }
    \end{tcolorbox}
    
    In the \href{https://github.com/sar-mo/CS2051-HonorsDiscreteMath/blob/main/sp23/hw-supplements/hw5-supp-slns/generate_schedule_sln.pdf}{solutions} for this supplement, we presented and implemented a successful algorithm for this problem. However, it was fairly inefficient, and could be much improved. Your task is as follows: \textbf{Describe an efficient solution for the \texttt{Task} \texttt{Scheduling} problem.}

    \vspace{2mm}
    A few notes:
    \begin{itemize}
        \item Consider how to best represent the input data, including possibly using external data structures.
        \item What is the Big-$\mathcal{O}$ time complexity of your solution? What about Big-$\Omega$? What input structures bottleneck your complexity?
    \end{itemize}

\pagebreak

\section*{Extra Problem [Spring 2022]}
    No one solved this problem last year, so we are posing it again.
    
    \vspace{2mm}
    Consider the grid shown below in Figure \ref*{E2}.
    
    \vspace{1mm}
    \setlength{\tabcolsep}{2.7pt}
    \begin{figure}[hbp]
        \centering
        \begin{tabular}{ccccccccccccc}
            & & & & & & S \\
            & & & & & S & A & S \\
            & & & & S & A & R & A & S \\
            & & & S & A & R & T & R & A & S \\
            & & S & A & R & T & H & T & R & A & S \\
            & S & A & R & T & H & A & H & T & R & A & S \\
            S & A & R & T & H & A & K & A & H & T & R & A & S \\
            & S & A & R & T & H & A & H & T & R & A & S \\
            & & S & A & R & T & H & T & R & A & S \\
            & & & S & A & R & T & R & A & S \\
            & & & & S & A & R & A & S \\
            & & & & & S & A & S \\
            & & & & & & S
        \end{tabular}
        \caption{A diamond-shaped, narcissistic grid of letters.}
        \label{E2}
    \end{figure}
    
    Suppose you start at any $S$, and can move only left, right, down, or up to adjacent letters. In how many ways can the palindrome SARTHAKAHTRAS be read if the same letter \underline{cannot} be used more than once in each sequence? \textbf{Prove your answer.}

    \vspace{2mm}
    \textit{(Hint): You can program a script to get your answer, and then work backwards to discover its origins.}


\end{document}